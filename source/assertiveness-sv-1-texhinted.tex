\documentclass[swedish,a4paper]{book}

% cd ~/Dropbox/Psykologprogrammet-tobias; biber --tool --validate_datamodel --fixinits --isbn13 --isbn-normalise --output-format=bibtex library.bib ; rsync library_bibertool.bib rstudio-tobias@kontor.hme.se:assert-yourself

% cd ~/Dropbox; rsync rstudio-tobias@kontor.hme.se:assert-yourself/assertiveness-sv-1-texhinted.pdf ./Konstruktiv-självhävdelse-arbetsbok_RÅTEXT.pdf

% https://www.quora.com/How-do-I-search-for-repeated-words-contained-within-a-text-file
% cat assertiveness-sv-1-texhinted.tex | tr '\n' ' ' | grep -o -E '\b((\w+)\s+)(\2|\1)+\b'

%%%%%%%%%%%%%%%%%%%%%%%%%%%%%%%%%%%%%%%%%%%%%%%%%%%%%%%%%%%%%
\setlength{\topskip}{\baselineskip}
\lineskiplimit = -10pt
\lineskip = 0pt
\frenchspacing
\usepackage[all]{nowidow}
\raggedbottom

%\usepackage[paperwidth=216mm,paperheight=216mm,heightrounded=true,twoside=true,left=36mm,textwidth=120mm]{geometry}
%\usepackage[heightrounded=true,showframe]{geometry}
%%%%%%%%%%%%%%%%%%%%%%%%%%%%%%%%%%%%%%%%%%%%%%%%%%%%%%%%%%%%%
\usepackage{adjustbox}
\usepackage{graphicx}
%%%%%%%%%%%%%%%%%%%%%%%%%%%%%%%%%%%%%%%%%%%%%%%%%%%%%%%%%%%%%

%http://tug.ctan.org/info/biblatex-cheatsheet/biblatex-cheatsheet.pdf
\usepackage{longtable,tabularx,booktabs,array,calc}
\usepackage{rotating}
\usepackage{afterpage}
\usepackage{amssymb,amsmath}
\usepackage{ifxetex,ifluatex}

\ifnum 0\ifxetex 1\fi\ifluatex 1\fi=0 % if pdftex
  \usepackage[T1]{fontenc}
  \usepackage[utf8]{inputenc}
\else % if luatex or xelatex
  \ifxetex
    \usepackage{mathspec}
  \else
    \usepackage{fontspec}
  \fi
  \defaultfontfeatures{Ligatures=TeX,Scale=MatchLowercase}
  %gm\setmainfont{Times New Roman}
  \setmainfont[Numbers=Lining]{Libre Baskerville}
  \newfontfamily{\sansserif}[
    Scale=MatchUppercase,
    BoldFont={Poppins Medium},
    ItalicFont={Poppins Italic},
    BoldItalicFont={Poppins Medium Italic}
    ]
    {Poppins Regular}
  \setsansfont[Scale=MatchUppercase]{Roboto} % Raleway
  % \setsansfont[
  %   Scale=MatchUppercase,
  %   BoldFont={Poppins Medium},
  %   ItalicFont={Poppins Italic},
  %   BoldItalicFont={Poppins Medium Italic}
  %   ]{Poppins Regular}
\fi
% use upquote if available, for straight quotes in verbatim environments
\IfFileExists{upquote.sty}{\usepackage{upquote}}{}
% use microtype if available
\IfFileExists{microtype.sty}{%
\usepackage[]{microtype}
\UseMicrotypeSet[protrusion]{basicmath} % disable protrusion for tt fonts
}{}
\PassOptionsToPackage{hyphens}{url} % url is loaded by hyperref
\usepackage[unicode=true]{hyperref}
\hypersetup{
            pdftitle={Respekt\textsuperscript{2} -- KBT för konstruktiv självhävdelse},
            pdfborder={0 0 0},
            breaklinks=true}
\urlstyle{same}  % don't use monospace font for urls
\ifnum 0\ifxetex 1\fi\ifluatex 1\fi=0 % if pdftex
  \usepackage[shorthands=off,main=swedish]{babel}
\else
   \usepackage{csquotes}
   \usepackage{polyglossia}
  \setmainlanguage[]{swedish}
  \setotherlanguage{english}
\fi
%%%%%%%%%%%%%%%%%% 
\usepackage[
  style=apa,
%  natbib=true,
  backend=biber,
  url=false,
  doi=false,
%  eprint=false
  indexing=true
]{biblatex}
\addbibresource{library_bibertool.bib}
%%%
\usepackage{filecontents}
\begin{filecontents*}{refs.bib}
@inbook{linehan1979assertion,
author = {Linehan, Marsha},
booktitle = {Cognitive-behavioral interventions: Theory, research, and procedures},
editor = {Kendall, Philip C and Hollon, Steven D},
pages = {205--237},
publisher = {Academic Press},
title = {{Structured cognitive-behavioural treatment of assertion problems}},
year = {1979}
}
\end{filecontents*}
\addbibresource{refs.bib}
%%%
\renewbibmacro*{citeindex}{\indexnames{labelname}{}}
%\usepackage{makeidx}
\usepackage{imakeidx}
\makeindex[
  title=Sak- och personregister,
  intoc,
  %options=-s mystyle
]
%%%%%%%%%%%%%%%%%% 
\linespread{1.125}
%%%%%%%%%%%%%%%%%% 
% Fix footnotes in tables (requires footnote package)
\IfFileExists{footnote.sty}{\usepackage{footnote}\makesavenoteenv{long table}}{}

\usepackage{parskip}[skip=.5\baselineskip]

\setlength{\emergencystretch}{3em}  % prevent overfull lines
\providecommand{\tightlist}{%
  \setlength{\itemsep}{0pt}\setlength{\parskip}{0pt}}
\setcounter{secnumdepth}{0}
% Redefines (sub)paragraphs to behave more like sections
\ifx\paragraph\undefined\else
\let\oldparagraph\paragraph
\renewcommand{\paragraph}[1]{\oldparagraph{#1}\mbox{}}
\fi
\ifx\subparagraph\undefined\else
\let\oldsubparagraph\subparagraph
\renewcommand{\subparagraph}[1]{\oldsubparagraph{#1}\mbox{}}
\fi

% set default figure placement to htbp
\makeatletter
\def\fps@figure{htbp}
\makeatother


%%%%%%%%%%%%%%%%%%%

\usepackage{multicol,enumitem}
%%%%%%%%%%%%%%%%%%%


\usepackage{tikz}

\tikzset{
    %Define style for boxes
    punkt/.style={
           circle,
           text width=.4\textwidth,
           minimum height=2em,
           inner sep=\baselineskip,
           outer sep=0,
           text centered},
}

\newcommand\nextup[1]{%
\clearpage\null
\vspace{0pt plus 1 fill}
\begin{center}%
\begin{tikzpicture}%
\node[
  punkt,
  draw=black,
  %fill=accent,
  %text=white
  ] {\sffamily #1};
\end{tikzpicture}%
\end{center}%
\vspace{0pt plus 3 fill}
\clearpage
}

%%%%%%%%%%%%%%%%%%%

\newcommand\xnum{4}
\newcommand\xlines[1]{%
\par\nobreak%
\renewcommand\xnum{#1}%
\vspace{1.5\baselineskip}%
\strut\begin{tikzpicture}%
\foreach \i in {1,...,\xnum} {%
\draw (0,2\baselineskip*\i-.4pt) -- (\textwidth,2\baselineskip*\i-.4pt);%
}%
\end{tikzpicture}\strut%
%\vspace*{.5\baselineskip}
}

%%%%%%%%%%%%%%%%%%%%%%%%%%%%%%%%%%%%%%%%%%%%%%%%%%%%%%%%%%%%%
\usepackage{fancyhdr}
\setlength{\headheight}{2\baselineskip}
\renewcommand{\headrulewidth}{0pt} % remove lines as well
\renewcommand{\footrulewidth}{0pt}

\pagestyle{fancy}
\renewcommand{\chaptermark}[1]{\markboth{#1}{}}
\renewcommand{\sectionmark}[1]{\markright{#1}}

\fancyhf{}
\fancyhead[LE,RO]{\sffamily\thepage}
\fancyhead[RE]{\sffamily\itshape{\nouppercase{\leftmark}}}
\fancyhead[LO]{\sffamily\itshape{\nouppercase{\rightmark}}}

\fancypagestyle{plain}{ %
  \fancyhf{} % remove everything
  \renewcommand{\headrulewidth}{0pt} % remove lines as well
  \renewcommand{\footrulewidth}{0pt}
}
%%%%%%%%%%%%%%%%%%%%%%%%%%%%%%%%%%%%%%%%%%%%%%%%%%%%%%%%%%%%%

%\definecolor{accent}{cmyk}{0, 0.7808, 0.4429, 0.1412}
%\definecolor{accent}{cmyk}{0,1,0,0}
\definecolor{accent}{cmyk}{0,0,0,.6}

%%%%%%%%%%%%%%%%%%%%%%%%%%%%%%%%%%%%%%%%%%%%%%%%%%%%%%%%%%%%%
% cat /usr/share/texlive/texmf-dist/tex/latex/base/book.cls
%%%%%%%%%%%%%%%%%%%%%%%%%%%%%%%%%%%%%%%%%%%%%%%%%%%%%%%%%%%%%
\usepackage{ragged2e} % to enable resetting back to justified text
%\newcommand\xspecialintrotext{}
%%%%%%%%%%%%%%%%%%%%%%%%%%%%%%%%%%%%%%%%%%%%%%%%%%%%%%%%%%%%%
\makeatletter
\renewcommand\tableofcontents{%
    \if@twocolumn
      \@restonecoltrue\onecolumn
    \else
      \@restonecolfalse
    \fi
    \chapter*{\sffamily\contentsname % Customized
        \@mkboth{%
           \contentsname}{\contentsname}}%
    \@starttoc{toc}%
    \if@restonecol\twocolumn\fi
    }
\renewcommand\chapter{\if@openright\cleardoublepage\else\clearpage\fi
                    \thispagestyle{plain}%
                    \global\@topnum\z@
                    \@afterindentfalse
                    \secdef\@chapter\@schapter}
\def\@chapter[#1]#2{\ifnum \c@secnumdepth >\m@ne
                       \if@mainmatter
                         \refstepcounter{chapter}%
                         \typeout{\@chapapp\space\thechapter.}%
                         \addcontentsline{toc}{chapter}%
                                   {\protect\numberline{\thechapter}#1}%
                       \else
                         \addcontentsline{toc}{chapter}{#1}%
                       \fi
                    \else
                      \addcontentsline{toc}{chapter}{#1}%
                    \fi
                    \chaptermark{#1}%
                    \addtocontents{lof}{\protect\addvspace{10\p@}}%
                    \addtocontents{lot}{\protect\addvspace{10\p@}}%
                    \if@twocolumn
                      \@topnewpage[\@makechapterhead{#2}]%
                    \else
                      \@makechapterhead{#2}%
                      \@afterheading
                    \fi}
\def\@makechapterhead#1{%
  \vspace*{0\p@}% % Customized, was 50
  {\parindent \z@ \raggedright \normalfont
    \ifnum \c@secnumdepth >\m@ne
      \if@mainmatter
        \Huge\sffamily\bfseries \textcolor{accent}{\@chapapp\space \thechapter} % Customized
        \par\nobreak
        %\vskip 20\p@ % Customized
        \normalsize % Customized
        \vskip 2\baselineskip % Customized
      \fi
    \fi
    \interlinepenalty\@M
    \Huge\sffamily\bfseries #1\par\nobreak
    %\vskip 40\p@
    \normalsize % Customized
    \vskip 3\baselineskip % Customized
  }}
\def\@schapter#1{\if@twocolumn
                   \@topnewpage[\@makeschapterhead{#1}]%
                 \else
                   \@makeschapterhead{#1}%
                   \@afterheading
                 \fi}
\def\@makeschapterhead#1{%
  \vspace*{0\p@}% % Customized, was 50
  {\parindent \z@ \raggedright
    \normalfont 
    \interlinepenalty\@M
    \Huge\sffamily\bfseries  #1\par\nobreak
    \normalsize % Customized
    \vskip 3\baselineskip % Customized
    %\xspecialintrotext{}
    %\renewcommand\xspecialintrotext{}
  }}
%%%%%%%%%%%%%%
% https://tex.stackexchange.com/questions/301531/place-text-on-top-of-bibliography-page
\defbibnote{wittyquote}{\justify\normalfont\normalsize Koncepten och strategierna i den här boken bygger på evidensbaserad psykologpraktik, primärt kognitiv beteendeterapi (KBT). KBT är ett slags psykoterapi som utgår från att hindrande, negativa känslor och beteenden är direkt påverkade av problematisk kognition, det vill säga tänkande.\par\vspace{.5\baselineskip}}



%%%%%%%%%%%%%% 
% https://tex.stackexchange.com/questions/36609/formatting-section-titles
%%%%%%%%%%%%%%
\renewcommand\section{\@startsection{section}{1}{\z@}%
                                   {1.25\baselineskip}%
                                   {.25\baselineskip}%
                                   {\fontsize{1.25\baselineskip}{1.25\baselineskip}\selectfont\sffamily\bfseries}} % Customized
\renewcommand\subsection{\@startsection{subsection}{1}{\z@}%
                                   {\baselineskip}%
                                   {.25\baselineskip}%
                                   {\fontsize{1\baselineskip}{1.25\baselineskip}\selectfont\sffamily\bfseries}} % Customized
\renewcommand\subsubsection{\@startsection{subsubsection}{1}{\z@}%
                                   {\baselineskip}%
                                   {.25\baselineskip}%
                                   {\fontsize{1\baselineskip}{1.25\baselineskip}\selectfont\sffamily\color{accent}\bfseries}} % Customized
%%%%%%%%%%%%%%
  \renewcommand\maketitle{\begin{titlepage}%
  \let\footnotesize\small
  \let\footnoterule\relax
  \let \footnote \thanks
  \null%\vfil % Customized
  %\vskip 60\p@ % Customized
  \begin{center}%
    {\Huge\sffamily\bfseries \@title \par}% % Customized
    \vskip 2\baselineskip% % Customized
    {\large\sffamily % Customized
     \lineskip .75em%
      \begin{tabular}[t]{c}%
        \@author
      \end{tabular}\par}%
      \vskip 1.5em%
        \null
  Svensk översättning och anpassning © 2019 Tobias Hagberg\par
  \vskip 1\baselineskip
  \textcolor{red}{OBS! Ej korrekturläst, granskad eller fullständigt indexerad version, får ej spridas}

    {\large \@date \par}%       % Set date in \large size.
  \end{center}\par
  \@thanks
  %\vfil\null
  \vfill
  % \centering
  %   \resizebox{5mm}{!}{
  %   \begin{tikzpicture}[line width=13mm,color=black,fill=black,x=5mm,y=5mm]
  %     %\clip (3,5) circle [radius=9.5];
  %     %\draw [fill] (3,5) circle [radius=9.5];
  %     \draw (0,0) -- (0,10);
  %     \draw (6,0) -- (6,10);
  %     \draw (0,4.25) -- (6,5.75);
  %   \end{tikzpicture}
  % }
  \end{titlepage}%
  }
%%%%%%%%%%%%%%

%%%%%%%%%%%%%%
\makeatother
%%%%%%%%%%%%%%%%%%%%%%%%%%%%%%%%%%%%%%%%%%%%%%%%%%%%%%%%%%%%%
% \usepackage{tocloft}
% %% Adjust sectional unit title fonts in ToC
% % http://ftp.acc.umu.se/mirror/CTAN/macros/latex/contrib/tocloft/tocloft.pdf
% \renewcommand{\cfttoctitlefont}{\sffamily\bfseries\Huge}
% \setlength\cftbeforetoctitleskip{0pt}
% \setlength\cftaftertoctitleskip{0pt}
% \setlength\cftaftertoctitle{0pt}
% \renewcommand{\cftchapfont}{\sffamily\bfseries}
% \renewcommand{\cftsecfont}{\sffamily}
% \renewcommand{\cftsubsecfont}{\sffamily}
% \renewcommand{\cftchappagefont}{\sffamily\bfseries}
% \renewcommand{\cftsecpagefont}{\sffamily}
% \renewcommand{\cftsubsecpagefont}{\sffamily}
%%%%%%%%%%%%%%%%%%%%%%%%%%%%%%%%%%%%%%%%%%%%%%%%%%%%%%%%%%%%%

\title{Respekt\textsuperscript{2} -- KBT för konstruktiv självhävdelse}
\date{}
\author{Fiona Michel \& Anthea Fursland}

\begin{document}

{
\maketitle
\thispagestyle{empty}
}
{
\setcounter{tocdepth}{1}
{%\sffamily
\tableofcontents}
%\pagestyle{plain}
}
%\pagestyle{plain}
\chapter{Hur hävdar man sig själv konstruktivt?}\label{chap1}

Du har säkert hört någon säga ''Du behöver stå upp för dig själv!'' Men
vad betyder det? Att hävda sig själv på ett konstruktivt sätt är ett
visst sätt att kommunicera. Det handlar om att uttrycka sina känslor,
tankar, uppfattningar och åsikter på ett öppet sätt, som samtidigt inte kränker
andra människors rättigheter\index{rättigheter}\index{kränkning}. Att vara aggressiv är en annan
kommunikationsstil, som kränker andras rättigheter. Ytterligare en
kommunikationsstil är den passiva, som är ett sätt att kommunicera som
inte respekterar dina egna rättigheter. Du har säkert också hört talas
om kommunikationsstilen passiv aggressivitet\index{passiv-aggressiv}. Det är när någon är
aggressiv, men på ett indirekt eller passivt sätt. Ett exempel är när
någon är arg men inte visar det öppet (som att skrika eller slåss), utan
istället kanske smäller igen dörren efter sig och drar sig undan.

\section{Myter om självhävdelse}\label{myter-om-sjuxe4lvhuxe4vdelse}

Det finns flera myter\index{myter} om självhävdelse. En del använder dessa som
ursäkter för att inte hävda sig själva bättre. Det är värt att titta
närmare på de här myterna i detalj. I kapitel~\ref{chap3}, som handlar om sätt att
tänka på ett mer självhävdande sätt, kommer vi fördjupa oss i olika sätt
att tänka som hindrar oss från att hävda oss själva. Dessutom kommer vi
att titta närmare på hur vi kan förändra det här sättet att tänka.

\subsection{Myt 1: ''Att hävda sig själv är att vara aggressiv''}\label{myt-1}

Vissa tror att de hävdar sig själva eftersom de uttrycker sina 
behov när de är aggressiva. Det är visserligen helt sant att både
aggressiv kommunikation och konstruktiv självhävdelse inbegriper att du
förmedlar vad du behöver. Skillnaden i hur du hävdar dina behov
aggressivt och hur du gör det konstruktivt är mycket viktig. Det handlar
bland annat om vilka ord, vilket tonläge och vilket kroppsspråk du
använder. Vi kommer att diskutera de här skillnaderna mer i detalj i
avsnitten om icke-verbala kännetecken för respektive
kommunikationsstil.

\begin{itemize}
\item
  Aggressiv kommunikation kränker andra personers rättigheter. De egna
  behoven har överhanden.\index{rättigheter}\index{kränkning!själv}
\item
  Konstruktivt självhävdande kommunikation respekterar både andra
  personers rättigheter och dina egna rättigheter.
\item
  Passiv kommunikation kränker dina egna rättigheter till förmån för
  andra personers behov.\index{kränkning!andra}
\end{itemize}

\subsection{Myt 2: ''Om jag hävdar mig själv får jag som jag vill''}\label{myt-2}

Att du hävdar dig själv konstruktivt betyder inte att du alltid får det
du vill. Att hävda sig själv på ett konstruktivt sätt ger faktiskt inga
garantier för några särskilda utfall alls. Att hävda sig själv handlar
om att uttrycka sig på ett sätt som är förenligt med både dina behov och
andras behov. Ibland betyder det att du får som du vill,
ibland att du inte alls får som du vill och ibland att du kommer att
lyckas förhandla fram en kompromiss som är tillfredsställande för alla parter.\index{kompromiss}

\subsection{Myt 3: ''Om jag ska hävda mig måste jag göra det jämt''}\label{myt-3}

Om du lär dig hur du kan hävda dig själv konstruktivt får du
valmöjligheten\index{flexibilitet} att kommunicera det som är viktigt för dig på ett sätt
som samtidigt respekterar andras rättigheter. Det betyder inte att du
måste stå upp för din åsikt i varje given situation. Kanske upptäcker du
att det i vissa situationer inte är det mest lämpliga sättet att agera.
Tänk dig till exempel att du är på en bar och plötsligt blir angripen
med aggressivitet och kanske till och med fysiskt våld. Genom att då stå upp
för dig själv skulle du kunna utsätta dig för risken att en inte helt
rationell person åsamkar dig skada. I det här fallet kan en passiv
inställning mycket väl vara den mest framgångsrika. Att lära sig
konstruktiv självhävdelse handlar om att ge sig själv valmöjligheter.

\section{Effekter av bristande konstruktiv
självhävdelse}\label{effekter-av-bristande-sjuxe4lvhuxe4vdelse}

Bristande konstruktiv självhävdelse föder låg självkänsla\index{självkänsla}. Om
du har för vana att kommunicera på ett passivt sätt säger du inte fullt
ut vad du verkligen känner och tänker\index{känslor}\index{tankar}. Det betyder att du riskerar att
böja dig för andras behov och önskemål. Det innebär att du bortser från dina egna behov och önskemål. Det kan ge en känsla av meningslöshet\index{meningslöshet} och av att du själv inte är den som styr i ditt liv.

Om du inte uttrycker dig öppet, utan gömmer dina tankar och känslor,
kan det få dig att känna dig spänd, stressad och orolig. Andra kan uppfatta dig som föraktfull.
Det kan också leda till destruktiva och nedbrytande relationer med
andra. Det kan leda till att det känns som att de som står dig närmast
inte egentligen känner dig.

Brist på självhävdelse är mycket vanligt bland dem med social ångest\index{social ångest}.
Personer med social ångest har en tendens att utgå från att andra personer är
dömande och kritiska mot dem, och undviker därför ofta sociala
situationer. Om du tror att du har social ångest, kan ett första steg vara att skaffa självhjälpsboken \citetitle{furmark2006social} av \textcite{furmark2006social}. Du kan använda den boken på egen hand, eller i terapi, med stöd av en KBT-psykolog eller -psykoterapeut.

Om du nästan alltid kommunicerar på ett aggressivt sätt kan du så
småningom förlora kontakten med vänner, och personer i din närhet kan tappa respekten för dig. Även det här kan leda till dålig självkänsla\index{självkänsla}.

Det finns mycket forskning om de negativa effekterna av bristande
konstruktiv självhävdelse -- det vill säga överskott\index{överskott} på passiv eller
aggressiv kommunikation. Personer som hävdar sig själva konstruktivt har
generellt sett mindre problem med depression\index{depression} och lever friskare liv.
Personer som har brister i självhävdelse löper större risk att hamna i
alkohol- eller drogmissbruk\index{alkoholmissbruk}\index{drogmissbruk}.

\section{Konstruktiv självhävdelse är något vi lär oss}

Förmågan att hävda sig själv är ett resultat av inlärning\index{inlärning}, både av sätt
att bete sig och sätt att tänka. Vi föds alla med en hög grad av
självhävdelse. Tänk på nyfödda som skriker när de behöver något --
bebisar uttrycker sina känslor utan spärrar. Med tiden anpassar de
gradvis sina beteenden, så att de ska passa med de svar de får från
omgivningen, det vill säga de responser de får från familjen och så
småningom från kompisar, auktoritetsfigurer, arbetskamrater och så
vidare. Om det till exempel var vanligt att i din familj hantera
konflikter med gormande och grälande, så kan du ha lärt dig att hantera
meningsskiljaktigheter på det sättet\index{konfliktlösning}. Du kan också ha svårt att hävda
dina behov om du under din uppväxt lärde dig att du alltid skulle tillfredsställa
andras behov, snarare än dina egna. Och om det i din
familj eller bland dina närmaste fanns en norm som sa att man inte skulle
uttrycka negativa känslor, så kanske du blev ignorerad eller förminskad om du ändå gjorde det. Då lärde du dig snabbt att inte uttrycka negativa
känslor.

Här är några frågor som kan vara bra att ställa till dig själv när du
tänker på hur du kan ha lärt dig att inte hävda dina behov.

\begin{itemize}
\item
  Hur hanterade din familj\index{familj} konflikter?\index{konfliktlösning}
\item
  Vad gjorde de när de inte höll med någon eller var arga på andra
  personer?
\item
  Hur lärde dina föräldrar dig att hantera konflikter?\index{konfliktlösning}
\item
  Vad förmedlade de för budskap till dig?
\item
  På vilket sätt lärde du dig att få det du ville ha, utan att fråga om
  det direkt? (Exempel på indirekta sätt kan vara att gråta, bråka, hota
  och så vidare.)
\item
  Använder du något av de sätten för att få som du vill idag?
\end{itemize}

Som du ser av exemplen ovan så finns det ofta begripliga och giltiga
skäl till att du har svårt för att hävda dig själv konstruktivt. Som
barn och tonåringar lärde vi oss att bete oss på sätt som fungerade för
oss i stunden. Om vi stod upp mot aggressiva föräldrar eller vänner, 
kan det ha lett till problem i de situationerna. Då lärde vi oss att hålla oss under
radarn istället. Eller så lärde vi oss att själva vara aggressiva, för att
överleva. Och det är troligt att de familjemedlemmar\index{familj} eller vänner
som lärde oss det här också själva i sin tur lärde sig det från andra.

Det är viktigt att du inte klandrar dig själv eller din familj för att
du kämpar med att hävda dig själv. Det kan vara hjälpsamt att tänka 
att du och din familj har varit fast i en ond spiral. Nu har du beslutat
dig för att bryta mönstret och lära dig nya sätt att tänka och nya sätt
att hävda dig själv konstruktivt. Det betyder att du inte kommer att
förmedla vidare de ohjälpsamma beteendena till dina egna
familjemedlemmar och vänner.

\section{Vad hindrar oss från att hävda
oss konstruktivt?}\label{vad-hindrar-oss-fruxe5n-att-huxe4vda-oss}

Flera olika saker kan hindra oss från att hävda oss själva på ett
konstruktivt sätt.

\subsection{Självbesegrande, hindrande
antaganden}\label{sjuxe4lvbesegrande-antaganden}

\index{antagande!hindrande|(}

Vi kan ha orealistiska föreställningar eller negativa uppfattningar om
vikten att hävda våra behov, vår förmåga att uttrycka våra behov eller
vad som kan hända om vi gör det. Det här är ofta en viktig orsak till
bristande självhävdelse. Exempel på sådana antaganden är:

\begin{itemize}
\item
  ''Om jag säger vad jag vill är jag oförskämd och självisk.''
\item
  ''Om jag säger vad jag vill bryr jag mig inte om vad andra vill.''
\item
  ''Om jag hävdar mina behov kommer andra att bli arga.''
\item
  ''Om jag hävdar mina behov kommer jag att förstöra relationer.''
\item
  ''Det är pinsamt att säga vad jag tänker.''
\end{itemize}

\index{antagande!hindrande|)}

I kapitel~\ref{chap3}, \textit{\nameref{chap3}}, finns fler exempel på ohjälpsamma
antaganden. Där får du också lära dig att tänka på ett mer konstruktivt
sätt.

\subsection{Behov av färdighetsträning}\label{fuxe4rdighetsbrister}

Det kan\index{färdigheter} också vara så att vi helt enkelt inte har tränat tillräckligt på
att hävda oss själva konstruktivt, verbalt och icke-verbalt. Vi kan kanske se
andra personer som är bra på att hävda sig själva och beundra deras
sätt att agera -- men själva har vi ingen aning om hur vi ska kunna göra samma sak. 
I kapitel~\ref{chap4}, \textit{\nameref{chap4}}, kommer vi att undersöka specifika tekniker
för konstruktiv självhävdelse.

\subsection{Ångest, stress och oro}\label{uxe5ngest-stress-och-oro}

Det\index{ångest}\index{stress}\index{oro} kan vara så att vi vet hur vi ska göra för att hävda oss, men att vi
blir så ångestfyllda att vi inte klarar av att utföra beteendet. Det kan
också handla om att vi blir så stressade att det blir svårt att både
tänka och handla. Då behöver vi lära oss hantera vår ångest och minska
den fysiska stressen i kroppen. Kapitel~\ref{chap5} kallas \textit{\nameref{chap5}} och handlar om
övningar som sänker den fysiska spänningsnivån i din kropp.

Om du brukar oroa dig mycket\index{generaliserat ångestsyndrom, GAD} kan du tackla det med en evidensbaserad
självhjälpsbok, exempelvis \citetitle{dugas2018} av \textcite{dugas2018}. Om du har så stark ångest att den resulterar i panikattacker\index{panikattack}, titta närmare på \citetitle{carlbring2013ingen} av \textcite{carlbring2013ingen}. Du kan använda böckerna på egen hand, eller i terapi, med stöd av en KBT-psykolog eller -psykoterapeut.

\subsection{Tolkningen av situationen}\label{tolkningen-av-situationen}

\index{tolkning}Det kan hända att vi inte riktigt kan välja vilka beteenden vi ska
använda i olika situationer. Det finns tre typer av misstag som
människor kan göra när de tolkar och utvärderar situationer. Vi kan
missförstå tydlig självhävdelse hos andra som aggression, vi kan missförstå brist
på självhävdelse hos andra som artighet eller så kan vi tänka att ''ligga lågt'' generellt sett är ett effektivt sätt att hantera situationer på. Du kommer att lära dig några tekniker för att ta itu med dessa hindrande tolkningarna i kapitel~\ref{chap3}, \textit{\nameref{chap3}}.

\subsection{Kulturella influenser och
generationsskillnader}\label{kulturella-influenser-och-generationsskillnader}

Vilka sätt vi agerar på kan också bero på starka kulturella\index{kultur|see{normer}}\index{normer} influenser
och generationsmässiga\index{generationsskillnad} skillnader i normer. Det är till exempel inte
lika uppskattat att hävda sig själv konstruktivt i vissa kulturer, som
det är i ett typiskt västerländsk sammanhang. Om du kommer från en sådan
miljö är det viktigt att noga överväga för- och nackdelarna med
självhävdande handlande. För dig kanske fördelarna med att leva i enlighet med dina kulturella normer överväger
fördelarna med konstruktiv självhävdelse i vissa situationer. Personer från äldre
generationer kan också tycka att det är svårt att hävda sig. De kan ha
vuxit upp i en tid då män fick lära sig att uttrycka känslor var detsamma som svaghet och kvinnor fick lära sig att det var aggressivt att
säga vad de tyckte eller behövde. Livslånga antaganden av det slaget kan
vara svåra att ändra, men det är möjligt!

\section{Hur bra är du på att hävda dig själv?}\label{hur-bra-uxe4r-du-puxe5-att-huxe4vda-dig-sjuxe4lv}

Det kan vara svårt att avgöra hur bra du är på att hävda dig själv. I
vissa situationer kanske du känner dig mycket kapabel, men i andra
situationer kan du märka att du faktiskt inte uttrycker hur du känner
eller tänker, eller blir upprörd eller frustrerad.

\subsection{Övning: skatta din förmåga till konstruktiv självhävdelse}\label{skatta}

I den här övningen får du hjälp att skatta hur bra du är på att hävda dig
själv konstruktivt i olika situationer. Det hjälper dig att identifiera de områden i livet som du kan utveckla din självhävdelseförmåga i. På sidan~\pageref{exercise1} finns en lista med olika situationer som kräver självhävdelse. Längst upp visas olika kategorier av
personer. Ta ruta för ruta och skatta varje kombination av situation och
personkategori. Till exempel kanske du tycker att det är relativt enkelt
att ge en främling en komplimang och sätter därför ''0'' i den rutan,
men du kanske också tycker att det är svårt att ge komplimanger till
personer med auktoritet, som din chef, och sätter därför ''4'' i den
rutan.

Fyll i varje ruta på en skala från 0 till 5. En skattning om ''0''
betyder att du utan problem kan hävda dig själv konstruktivt i den givna situationen.
En skattning om ''5'' betyder att du inte alls kan hävda dig själv konstruktivt i
den situationen.

Behåll en kopia av dina svar. Du kommer använda den
igen i kapitel~\ref{chap10}, \textit{\nameref{chap10}} för att utvärdera om och hur mycket du
har blivit bättre på att hävda dig själv konstruktivt.

\section{Sammanfattning}\label{sammanfattning}

\begin{itemize}
\item
  Konstruktiv självhävdelse är ett sätt att kommunicera som innebär att
  du uttrycker dina behov, åsikter och känslor, samtidigt som du
  respekterar andras rättigheter. Det skiljer sig från aggressivt
  beteende, som kränker andras rättigheter, och passivt beteende, som
  kränker dina egna rättigheter.
\item
  Även om vi har förmågan att hävda oss själva i de flesta situationer,
  kan det i vissa situationer vara svårt för oss att göra det.
\item
  Bristande konstruktiv självhävdelse kan leda till dålig självkänsla.
\item
  Vi kommer alla till världen med full förmåga till självhävdelse, men
  när vi växer upp lär vi oss också andra kommunikationsstilar.
\item
  Miljön vi befinner oss i kan göra det svårare för oss att hävda oss
  själva konstruktivt.
\item
  Ibland har vi hindrande föreställningar och antaganden\index{antagande!hindrande} om oss själva,
  andra människor och världen i stort, som gör det svårt för oss att hävda oss
  själva.
\end{itemize}


\newlength\xwidth
\newlength\xheight
%\setlength\xwidth{\textheight}
%\setlength\xheight{\textwidth}


\newcommand\assessment[2]{%
\begin{sidewaystable}%
\setlength\xwidth{570pt}
\setlength\xheight{570pt}
\begin{longtable}[]{rccccccc}
\toprule
\begin{minipage}[c][.06\xheight]{0.15\xwidth}\raggedright\strut
\strut
\end{minipage} & \begin{minipage}[c][.06\xheight]{0.09\xwidth}\centering\strut
{\sffamily\bfseries Vänner av samma kön}\strut
\end{minipage} & \begin{minipage}[c][.06\xheight]{0.09\xwidth}\centering\strut
{\sffamily\bfseries Vänner av annat kön}\strut
\end{minipage} & \begin{minipage}[c][.06\xheight]{0.09\xwidth}\centering\strut
{\sffamily\bfseries Auktoriteter}\strut
\end{minipage} & \begin{minipage}[c][.06\xheight]{0.09\xwidth}\centering\strut
{\sffamily\bfseries Främlingar}\strut
\end{minipage} & \begin{minipage}[c][.06\xheight]{0.09\xwidth}\centering\strut
{\sffamily\bfseries Arbets-\\kollegor}\strut
\end{minipage} & \begin{minipage}[c][.06\xheight]{0.09\xwidth}\centering\strut
{\sffamily\bfseries Kärleks-\\relationer}\strut
\end{minipage} & \begin{minipage}[c][.06\xheight]{0.09\xwidth}\centering\strut
{\sffamily\bfseries Butiks-\\personal}\strut
\end{minipage}\tabularnewline
\midrule
\endhead
\begin{minipage}[c][.06\xheight]{0.15\xwidth}\flushright\strut
{\sffamily\bfseries Säga nej}\strut
\end{minipage} & \begin{minipage}[t]{0.09\xwidth}\raggedright\strut
\strut
\end{minipage} & \begin{minipage}[t]{0.09\xwidth}\raggedright\strut
\strut
\end{minipage} & \begin{minipage}[t]{0.09\xwidth}\raggedright\strut
\strut
\end{minipage} & \begin{minipage}[t]{0.09\xwidth}\raggedright\strut
\strut
\end{minipage} & \begin{minipage}[t]{0.09\xwidth}\raggedright\strut
\strut
\end{minipage} & \begin{minipage}[t]{0.09\xwidth}\raggedright\strut
\strut
\end{minipage} & \begin{minipage}[t]{0.09\xwidth}\raggedright\strut
\strut
\end{minipage}\tabularnewline
\midrule
\begin{minipage}[c][.06\xheight]{0.15\xwidth}\flushright\strut
{\sffamily\bfseries Ge komplimang}\strut
\end{minipage} & \begin{minipage}[t]{0.09\xwidth}\raggedright\strut
\strut
\end{minipage} & \begin{minipage}[t]{0.09\xwidth}\raggedright\strut
\strut
\end{minipage} & \begin{minipage}[t]{0.09\xwidth}\raggedright\strut
\strut
\end{minipage} & \begin{minipage}[t]{0.09\xwidth}\raggedright\strut
\strut
\end{minipage} & \begin{minipage}[t]{0.09\xwidth}\raggedright\strut
\strut
\end{minipage} & \begin{minipage}[t]{0.09\xwidth}\raggedright\strut
\strut
\end{minipage} & \begin{minipage}[t]{0.09\xwidth}\raggedright\strut
\strut
\end{minipage}\tabularnewline
\midrule
\begin{minipage}[c][.06\xheight]{0.15\xwidth}\flushright\strut
{\sffamily\bfseries Uttrycka en åsikt}\strut
\end{minipage} & \begin{minipage}[t]{0.09\xwidth}\raggedright\strut
\strut
\end{minipage} & \begin{minipage}[t]{0.09\xwidth}\raggedright\strut
\strut
\end{minipage} & \begin{minipage}[t]{0.09\xwidth}\raggedright\strut
\strut
\end{minipage} & \begin{minipage}[t]{0.09\xwidth}\raggedright\strut
\strut
\end{minipage} & \begin{minipage}[t]{0.09\xwidth}\raggedright\strut
\strut
\end{minipage} & \begin{minipage}[t]{0.09\xwidth}\raggedright\strut
\strut
\end{minipage} & \begin{minipage}[t]{0.09\xwidth}\raggedright\strut
\strut
\end{minipage}\tabularnewline
\midrule
\begin{minipage}[c][.06\xheight]{0.15\xwidth}\flushright\strut
{\sffamily\bfseries Be om hjälp}\strut
\end{minipage} & \begin{minipage}[t]{0.09\xwidth}\raggedright\strut
\strut
\end{minipage} & \begin{minipage}[t]{0.09\xwidth}\raggedright\strut
\strut
\end{minipage} & \begin{minipage}[t]{0.09\xwidth}\raggedright\strut
\strut
\end{minipage} & \begin{minipage}[t]{0.09\xwidth}\raggedright\strut
\strut
\end{minipage} & \begin{minipage}[t]{0.09\xwidth}\raggedright\strut
\strut
\end{minipage} & \begin{minipage}[t]{0.09\xwidth}\raggedright\strut
\strut
\end{minipage} & \begin{minipage}[t]{0.09\xwidth}\raggedright\strut
\strut
\end{minipage}\tabularnewline
\midrule
\begin{minipage}[c][.06\xheight]{0.15\xwidth}\flushright\strut
{\sffamily\bfseries Uttrycka ilska}\strut
\end{minipage} & \begin{minipage}[t]{0.09\xwidth}\raggedright\strut
\strut
\end{minipage} & \begin{minipage}[t]{0.09\xwidth}\raggedright\strut
\strut
\end{minipage} & \begin{minipage}[t]{0.09\xwidth}\raggedright\strut
\strut
\end{minipage} & \begin{minipage}[t]{0.09\xwidth}\raggedright\strut
\strut
\end{minipage} & \begin{minipage}[t]{0.09\xwidth}\raggedright\strut
\strut
\end{minipage} & \begin{minipage}[t]{0.09\xwidth}\raggedright\strut
\strut
\end{minipage} & \begin{minipage}[t]{0.09\xwidth}\raggedright\strut
\strut
\end{minipage}\tabularnewline
\midrule
\begin{minipage}[c][.06\xheight]{0.15\xwidth}\flushright\strut
{\sffamily\bfseries Uttrycka ömhet}\strut
\end{minipage} & \begin{minipage}[t]{0.09\xwidth}\raggedright\strut
\strut
\end{minipage} & \begin{minipage}[t]{0.09\xwidth}\raggedright\strut
\strut
\end{minipage} & \begin{minipage}[t]{0.09\xwidth}\raggedright\strut
\strut
\end{minipage} & \begin{minipage}[t]{0.09\xwidth}\raggedright\strut
\strut
\end{minipage} & \begin{minipage}[t]{0.09\xwidth}\raggedright\strut
\strut
\end{minipage} & \begin{minipage}[t]{0.09\xwidth}\raggedright\strut
\strut
\end{minipage} & \begin{minipage}[t]{0.09\xwidth}\raggedright\strut
\strut
\end{minipage}\tabularnewline
\midrule
\begin{minipage}[c][.06\xheight]{0.15\xwidth}\flushright\strut
{\sffamily\bfseries Värna rättigheter och uttrycka behov}\strut
\end{minipage} & \begin{minipage}[t]{0.09\xwidth}\raggedright\strut
\strut
\end{minipage} & \begin{minipage}[t]{0.1\xwidth}\raggedright\strut
\strut
\end{minipage} & \begin{minipage}[t]{0.09\xwidth}\raggedright\strut
\strut
\end{minipage} & \begin{minipage}[t]{0.09\xwidth}\raggedright\strut
\strut
\end{minipage} & \begin{minipage}[t]{0.09\xwidth}\raggedright\strut
\strut
\end{minipage} & \begin{minipage}[t]{0.09\xwidth}\raggedright\strut
\strut
\end{minipage} & \begin{minipage}[t]{0.09\xwidth}\raggedright\strut
\strut
\end{minipage}\tabularnewline
\midrule
\begin{minipage}[c][.06\xheight]{0.09\xwidth}\flushright\strut
{\sffamily\bfseries Ge kritik}\strut
\end{minipage} & \begin{minipage}[t]{0.15\xwidth}\raggedright\strut
\strut
\end{minipage} & \begin{minipage}[t]{0.09\xwidth}\raggedright\strut
\strut
\end{minipage} & \begin{minipage}[t]{0.09\xwidth}\raggedright\strut
\strut
\end{minipage} & \begin{minipage}[t]{0.09\xwidth}\raggedright\strut
\strut
\end{minipage} & \begin{minipage}[t]{0.09\xwidth}\raggedright\strut
\strut
\end{minipage} & \begin{minipage}[t]{0.09\xwidth}\raggedright\strut
\strut
\end{minipage} & \begin{minipage}[t]{0.09\xwidth}\raggedright\strut
\strut
\end{minipage}\tabularnewline
\midrule
\begin{minipage}[c][.06\xheight]{0.15\xwidth}\flushright\strut
{\sffamily\bfseries Ta emot kritik}\strut
\end{minipage} & \begin{minipage}[t]{0.09\xwidth}\raggedright\strut
\strut
\end{minipage} & \begin{minipage}[t]{0.09\xwidth}\raggedright\strut
\strut
\end{minipage} & \begin{minipage}[t]{0.09\xwidth}\raggedright\strut
\strut
\end{minipage} & \begin{minipage}[t]{0.09\xwidth}\raggedright\strut
\strut
\end{minipage} & \begin{minipage}[t]{0.09\xwidth}\raggedright\strut
\strut
\end{minipage} & \begin{minipage}[t]{0.09\xwidth}\raggedright\strut
\strut
\end{minipage} & \begin{minipage}[t]{0.09\xwidth}\raggedright\strut
\strut
\end{minipage}\tabularnewline
\midrule
\begin{minipage}[c][.06\xheight]{0.15\xwidth}\flushright\strut
{\sffamily\bfseries Inleda en konversation}\strut
\end{minipage} & \begin{minipage}[t]{0.09\xwidth}\raggedright\strut
\strut
\end{minipage} & \begin{minipage}[t]{0.09\xwidth}\raggedright\strut
\strut
\end{minipage} & \begin{minipage}[t]{0.09\xwidth}\raggedright\strut
\strut
\end{minipage} & \begin{minipage}[t]{0.09\xwidth}\raggedright\strut
\strut
\end{minipage} & \begin{minipage}[t]{0.09\xwidth}\raggedright\strut
\strut
\end{minipage} & \begin{minipage}[t]{0.09\xwidth}\raggedright\strut
\strut
\end{minipage} & \begin{minipage}[t]{0.09\xwidth}\raggedright\strut
\strut
\end{minipage}\tabularnewline
\bottomrule
\end{longtable}
\label{#1}
\end{sidewaystable}%
}

\assessment{exercise1}

\nextup{I nästa kapitel går genom vad som kännetecknar de aggressiva, de
konstruktivt självhävdande och de passiva kommunikationsstilarna. Varje
stil har sina egna förtjänster och kostnader.}

\chapter{Skillnader mellan olika kommunikationsstilar}\label{chap2}

\section{Det är skillnad på verbal och icke-verbal kommunikation}

Det är viktigt att du lär dig skilja på verbala och icke-verbala kännetecken för de olika kommunikationsstilarna. När vi känner till dessa kan vi lättare upptäcka passiva, självhävdande och aggressiva beteenden både hos oss själva och hos andra.

Första steget för att förändra ett beteende är att känna igen de delar som behöver ändras. Du kanske redan är bra på att verbalt ta upp viktiga saker på ett självhävdande, konstruktivt sätt, men åtföljt av icke-verbal kommunikation\index{icke-verbal kommunikation}\index{verbal kommunikation}\index{kommunikation!icke-verbal}\index{kommunikation!verbal} som är passiv och därför säger emot dina verbala budskap. Om du till exempel verbalt hävdar dig själv med att säga ''Jag tycker inte om när du gör så där'', men gör det med mycket svag röst, utan ögonkontakt och medan du skruvar obekvämt på dig -- då kommer ditt icke-verbala beteende att underminera ditt verbala budskap. Resultatet kan bli att mottagaren inte tar budskapet på allvar. 

Du kommer att märka att varje sätt att kommunicera både ger utdelning och är förknippat med kostnader\index{fördelar}\index{nackdelar}. Det är viktigt att erkänna att även självhävdande, konstruktivt beteende har ett pris. Till exempel kan priset av att du börjar bete dig självhävdande vara att andra människor omkring dig inte tycker om det, och visar det, eftersom de kan ha vant sig vid att du tidigare böjde dig för deras behov och önskemål. Om du är medveten om den möjligheten kan det kanske göra det lättare för dig att förändra ditt beteende.

När du läser beskrivningarna nedan, tänk på vilka beteenden du själv saknar och vill lägga till din beteenderepertoar!

\pagebreak[4]\section{Passiv kommunikation}

\index{kännetecken!passivitet|(}

\subsection{Definition}

\begin{itemize}

\item Att du inte uttrycker dina ärliga känslor, tankar och övertygelser. Det innebär att du därmed tillåter andra att kränka dina rättigheter. Det kan också betyda att du uttrycker dina tankar och känslor på ett ursäktande, självutplånande sätt -- så att andra enkelt kan bortse från dem.

\item Att du kränker dina egna rättigheter.

\item Ibland även att du visar en subtil avsaknad av respekt för andra människors förmåga att klara av besvikelser, axla ansvar eller hantera sina egna problem.

\end{itemize}

\subsection{Verbala kännetecken}

\begin{itemize}

\item långa, virriga meningar

\item tassande som katten kring het gröt

\item trevande tal, fyllt av pauser

\item frekvent harklande

\item ursäktande utan anledning, i en mjuk och ostadig röst

\item användande av meningar som ''om det inte vore till för mycket besvär \ldots{}''

\item många fyllnadsord som ''kanske'', ''hmm'', ''du vet''

\item ofta enformig och monoton röst

\item ibland gnällig eller retande ton

\item överdrivet varm eller överdrivet mjuk röst

\item ofta lämnas resonemang oavslutade

\item ideliga rättfärdiganden, exempelvis ''jag brukar normalt inte säga någonting''

\item ursäkter, exempelvis ''jag är hemskt ledsen att jag stör''

\item upphävande av det egna talet, exempelvis ''det är bara min åsikt'' eller ''jag kan ha fel''

\item själv-avfärdande, som ''det är inte viktigt'' eller ''det spelar egentligen ingen roll''

\item själv-nedsättande tal, som ''jag är hopplös \ldots{}'' eller ''du känner mig \ldots{}''

\end{itemize}

\subsection{Icke-verbala kännetecken}

\begin{itemize}

\item undviker ögonkontakt

\item tittar ned

\item ibland hopsjunken hållning

\item vridandes på händerna

%\item uttrycker ilska med skratt eller blinkningar 

\item uttrycker ilska eller tar emot kritik med ''spökleende''

\item skakande käkar

\item bitande i läpparna

\end{itemize}

\subsection{Kognitiv stil (tänkande)}

\begin{itemize}

\item ''Jag räknas inte''

\item ''Mina känslor, behov och tankar är mindre viktiga än dina''

\item ''Andra personer kommer tänka illa om mig eller inte tycka om mig''

\item ''Om jag säger nej kan någon blir upprörd, om det händer är det mitt fel''

\end{itemize}

\subsection{Fördelar}

\begin{itemize}

\item Får beröm för att vara en osjälvisk, hygglig person

\item Får sällan skulden när något går fel, eftersom du sällan tar initiativ

\item Andra skyddar och tar hand om dig

\item Undviker, skjuter upp eller gömmer konflikter, för att kortsiktigt minska ångest och obehag

\end{itemize}

\subsection{Kostnader}

\begin{itemize}

\item Ibland benägenhet att bygga upp stress och ilska som kan explodera på ett väldigt aggressivt sätt

\item Andra personer kan ofta komma att ställa orimliga krav på dig

\item Kan fastna i relationer som inte är konstruktiva och som är svåra att ändra

\item Begränsar självbilden till att passa andras bild av dig som en älskvärd person

\item Samtidigt som du trycker undan ilska och frustration minskar det benägenheten att känna positiva känslor

\item Försämrad självkänsla

\end{itemize}

\index{kännetecken!passivitet|)}

\section{Aggressiv kommunikation}

\index{kännetecken!aggressivitet|(}

\subsection{Definition}

\begin{itemize}

\item Du står upp för dina egna rättigheter och uttrycker dina tankar, känslor och övertygelser men på ett sätt som oftast inte är passande och som kränker den andra personens rättigheter.

\item Andra människor känner olust i ett möte med en aggressiv person.

\item Du behåller övertaget genom att du trycker ned andra.

\item Du attackerar när du känner dig hotad.

\end{itemize}

\subsection{Verbala kännetecken}

\begin{itemize}

%\item Gällt, sarkastiskt eller nedlåtande röstläge

\item Ohämmat tal, utan tvekan

\item Ofta avsnäsande och vasst avklippt tal

\item Ofta snabbt tal

\item Tonvikten läggs på nedlåtande ord

\item Fast, stark röst

\item Tonen sarkastisk, kall och hård

\item Ofta gäll och ibland skrikande röst som stiger mot slutet

\item Användande av hot, exempelvis ''Det är bäst att du passar dig \ldots{}'' eller ''Om du inte gör som jag säger så \ldots{}''

\item Förnedrande tal, exempelvis ''Du måste skoja med mig \ldots{}'' eller ''Var inte så löjlig \ldots{}''

\item Värderande kommentarer, med tonvikten lagd på ord som ''borde'', ''dålig'', ''skulle ha'' och så vidare

\item Sexistiska och rasistiska kommentarer

\item Skryt, exempelvis ''Jag har inte sådana problem som du har''

\item Åsikter uttrycks som fakta, exempelvis ''Ingen vill bete sig på det sättet'' eller ''Det är ett värdelöst sätt att göra det på''

\item Hotfulla frågor och förminskande kommentarer, exempelvis ''Men är du inte klar med det där ännu?'' eller ''Varför i hela friden gör du det där?''

\end{itemize}

\subsection{Icke-verbala kännetecken}

\begin{itemize}

\item Intrång i den andra personens privata sfär

\item Den andra personen stirras ut

\item Gester som pekande och knuten näve

\item Rastlöst travande fram och tillbaka

\item Korsade armar (onåbar)

\item Leenden som kan bli hånfulla

\item Rynkade ögonbryn 

\item Sammanbiten käke

\end{itemize}

\subsection{Kognitiv stil (tänkande)}

\begin{itemize}

\item ''Jag tar dig innan du får en chans att ta mig.''

\item ''Jag ska vara först.''

\item ''Världen är ett slagfält och jag är här för att segra.''

\end{itemize}

\subsection{Fördelar}

\begin{itemize}

\item Du kan få andra personer att göra saker åt dig som du själv egentligen är ansvarig för

\item Saker tenderar att gå din väg

\item Du är mindre sårbar

\item Du känner att du har kontroll

\item Du får utlopp för spänning

\item Du känner dig stark

\end{itemize}

\subsection{Kostnader}

\begin{itemize}

\item Ditt beteende kommer att ge dig fiender och skapa förakt bland dem omkring dig.

\item Du kan känna dig paranoid och bli rädd att förlora din position.

\item Om du alltid försöker kontrollera andra kan det bli svårt för dig att slappna av.

\item Dina relationer tenderar att bli baserade på negativa känslor och blir sannolikt mer instabila.

\item Aggressiva personer tenderar att känna sig underlägsna inombords och försöker kompensera för det genom att trycka ned andra.

\item Känslor av skuld och skam

\item Minskat självförtroende och försämrad självkänsla

\end{itemize}

\index{kännetecken!aggressivitet|)}

\section{Självhävdande kommunikation}

\index{kännetecken!självhävdelse|(}

\subsection{Definition}

\begin{itemize}

\item Ett sätt att kommunicera känslor, tankar och övertygelser på ett öppet och ärligt sätt, som inte kränker andras rättigheter

\item Det är ett alternativ till att vara aggressiv och att kränka andras rättigheter, och till att vara passiv och kränka dina egna rättigheter

\end{itemize}

\subsection{Verbala kännetecken}

\begin{itemize}

\item Stadig, avslappnad röst

\item Flytande tal, utan tvekan

\item Stadig och jämn rytm

\item Tonen är i mellanregistret, rik och varm

\item Uppriktighet och tydlighet

\item Rösten varken överdrivet högljudd eller tyst

\item Rösten är anpassad i styrka för situationen

\item Jag-budskap (''Jag tycker om'', ''Jag vill'', ''Jag tycker inte om'') som är kortfattade och kärnfulla

\item Samarbetsfraser, som ''Vad tänker du om det här?''

\item Engagerade uttryck av intresse, som ''Jag skulle vilja \ldots{}''

\item Åtskillnad mellan fakta och åsikter, exempelvis ''Jag upplever det på ett annat sätt''

\item Förslag utan ''borden'' eller ''måsten'', exempelvis ''Hur skulle det vara om \ldots{}'' eller ''Skulle du vilja \ldots{}''

\item Konstruktiv kritik utan skuldbeläggande, exempelvis ''Jag känner mig irriterad när du avbryter mig''

\item Efterfrågande av andras åsikter, exempelvis ''Hur passar det här in med dina förslag?''

\item Villighet att utforska andra lösningar, exempelvis ''Hur kan vi komma runt det här problemet?''

\end{itemize}

\subsection{Icke-verbala kännetecken}

\begin{itemize}

\item Aktivt lyssnande

\item Direkt ögonkontakt, utan att stirra

\item Upprätt, balanserad och öppen kroppshållning

\item Öppna handrörelser

\item Leenden när du är tillfreds

\item Rynkad panna när du är arg

\item Lugna ansiktsdrag

\item Avslappnad käke

\end{itemize}

\subsection{Kognitiv stil (tänkande)}

\begin{itemize}

\item ''Jag kommer inte låta dig dra fördel av mig och jag kommer inte attackera dig för att du är den du är.''

\end{itemize}

\subsection{Fördelar}

\begin{itemize}

\item Ju mer du står upp för dig själv och ju mer du agerar på ett sätt som du själv respekterar, desto bättre självkänsla får du.

\item Dina chanser att saker går din väg i livet ökar dramatiskt.

\item Om du uttrycker dig själv när du känner, tänker eller behöver något bygger du inte upp bitterhet mot andra.

\item Om du drivs mindre av behovet att skydda dig själv, så kan du se, höra och älska andra mycket lättare.

\end{itemize}

\subsection{Kostnader}

\begin{itemize}

\item Vänner och familj kan ha tjänat på att du har agerat passivt och kan sabotera din nya konstruktiva, självhävdande kommunikationsstil.

\item Du kanske går emot antaganden och värden som du har haft sedan barndomen, och det kan vara skrämmande för dig själv.

\item Det finns inga garantier för några särskilda utfall.

\item Att hävda sig själv på ett konstruktivt sätt är ofta förenat med smärta.

\end{itemize}

\index{kännetecken!självhävdelse|)}

\section{Sammanfattning}

\begin{itemize}

\item Det finns avgörande verbala och icke-verbala skillnader mellan aggressiv, självhävdande och passiv kommunikation. Varje stil har sina egna fördelar och kostnader.

\end{itemize}

\nextup{I nästa kapitel tittar vi närmare på vanliga tankemönster, som kan hindra oss från att hävda oss själva. Vi lär oss också hur vi kan balansera dessa med konstruktivt tänkande.}

\chapter{Sätt att tänka för konstruktiv självhävdelse}\label{chap3}

\section{Tänkandet vid bristande självhävdelse}

\index{antagande!hindrande|(}

Vårt sätt att tänka kan hindra oss från att hävda oss själva på ett konstruktivt sätt, som vi konstaterade i kapitel~\ref{chap1}. Alla har vi antaganden och föreställningar om oss själva, andra människor och om hur världen fungerar. Dessa antaganden är ett resultat av de upplevelser i tillvaron som vi har haft, och är förstås helt logiska utifrån hur vi förstod världen i de situationerna. Det kan dock hända att vi inte har uppdaterat vårt tänkande, fast vi har rört på oss och lämnat de upplevelserna bakom oss. Kanske har vi till exempel fått lära oss av våra familjer att vi riskerar att bli förlöjligade om vi visar att vi är ledsna. Då är det inte så konstigt om det har resulterat i antagandet ''att uttrycka ledsenhet är dåligt, om jag gör det kommer andra att se ned på mig''. Även om vi har blivit äldre och inte längre befinner oss i den ursprungliga situationen, kan antagandet finnas kvar. Vi fortsätter att utgå från att våra antaganden är sanna, utan att testa sanningshalten i dem. Ett resultat av det kan vara att vi inte uttrycker vår ledsenhet när vi har den, och istället blir stressade eller deprimerade. %Det kan vara som om vi inte står i kontakt med oss själva, eller världen. 
Eller så uttrycker vi vår ledsenhet som ilska mot oss själva eller andra människor.

Här är en lista med tänkande som typiskt sett inte är konstruktivt självhävdande. Se om några av exemplen stämmer in på dig.

\begin{itemize}

\item ''Jag borde inte säga vad jag verkligen känner eller tänker eftersom jag inte vill tynga ned andra med mina problem.''

\item ''Om jag hävdar mig själv kommer den andra personen att bli upprörd och det kommer förstöra vår relation.''

\item ''Det blir bara hemskt pinsamt om jag säger vad jag tänker.''

\item ''Om någon säger 'nej' till min begäran är det för att personen inte tycker om mig eller älskar mig.''

\item ''Jag borde inte behöva säga vad jag behöver eller hur jag känner: Människor som står nära mig borde redan veta vad det är.''

\item ''Att säga vad man tycker visar att man inte bryr sig om andra, det är oförskämt och själviskt.''

\item ''Jag har ingen rätt att ändra mig i en fråga; det har inte andra heller.''

\item ''Det löser sig på något sätt ändå, och hur som helst är det inte mitt fel.''

\item ''Folk ska hålla sina känslor för sig själva.''

\item ''Om jag visar att jag har ångest kommer andra människor tycka att jag är svag, och förlöjliga eller utnyttja mig.''

\item ''Om jag tar emot och accepterar en komplimang från någon betyder det att jag förhäver mig själv.''

\end{itemize}

Ägna en minut åt att försöka identifiera några ytterligare hindrande antaganden som du har. Kapitel~\ref{chap6}, \ref{chap7} och \ref{chap8} handlar om hur du kan identifiera och hantera tänkande som inte möjliggör konstruktiv självhävdelse när du behöver säga nej, ta och ge kritik och hantera besvikelser. Kanske kan du få några ytterligare uppslag från de kapitlen?

\xlines{7}

\index{antagande!hindrande|)}

\section{Allas rätt att hävda sig själva}

\index{rättigheter!allas|(}

Många av idéerna som idag förknippas med träning i konstruktiv självhävdelse kommer från \begin{english}\citetitle{smith1975say}\end{english} av \textcite{smith1975say}. Boken innehåller en  ''rättighetsförklaring''. Varje människa har rätt att hävda sig själv. Här är några av rättigheterna som du och alla andra har.

\begin{itemize}

\item Du har rätt att vara domare över ditt eget beteende, dina egna tankar och dina egna känslor, och att ta ansvar för deras uttryck och konsekvenser.

\item Du har rätt att säga 'nej'.

\item Du har rätt att inte ange några skäl eller ursäkter för ditt beteende.

\item Du har rätt att själv avgöra om du är ansvarig för att hitta lösningar på andra människors problem.

\item Du har rätt att ändra åsikt.

\item Du har rätt att ha en annan åsikt än andra personer.

\item Du har rätt att göra misstag -- och att ta ansvar för dem.

\item Du har rätt att säga ''Jag vet inte''.

\item Du har rätt att vara ologisk när du fattar beslut.

\item Du har rätt att säga ''Jag förstår inte''.

\item Du har rätt att säga ''Jag bryr mig inte''.

\end{itemize}

En viktig del av dessa \textbf{rättigheter} är att de är sammanlänkade med \textbf{ansvar}\index{ansvar}. Den första punkten var som du såg att du har rätt till dina egna tankar, beteenden och känslor, men också att det är du som behöver ta ansvar för konsekvenserna av ditt tänkande och handlande. Ibland tror människor att de hävdar sig själva konstruktivt, fast de i själva verket glömmer bort att uppmärksamma vilka konsekvenser som deras handlande får på andra personers rättigheter. Det sättet att handla ligger närmare en aggressiv kommunikationsstil.

Försök komma på ytterligare några rättigheter, särskilt sådana som balanserar de ohjälpsamma, hindrande antagandena som du skrev upp i förra avsnittet.

\xlines{7}

\index{rättigheter!allas|)}

\section{Balansera dina hindrande antaganden}

\index{tankedagbok|(}
\index{antagande!hindrande!balansera|(}
\index{balansera antagande|(}

\subsection{Tankedagbok}

Utan en stöttande struktur kan det vara svårt att balansera och utveckla sitt tänkande. Det är ofta svårt att hålla alla beskrivningar och all motsägelsefull information i huvudet. Det bästa sättet att hantera det problemet är att skriva ned alla aspekter i en tankedagbok. För att visa vad det är för något, och hur man kan använda den, har vi fyllt i ett exempelformulär med svar på de frågor som du behöver ställa dig själv för att komma på en mer balanserad tanke. Sedan får du ett tomt formulär som du kan använda för att balansera en egen hindrande tanke.

%Tankedagboken frågar dig först vilken negativ tanke som du har. 
Börja med att \textbf{identifiera och beskriva situationen} som du var i när du tänkte tanken. I det följande exemplet skulle situationen kunna beskrivas som ''Jag frågade min vän om hon ville gå och shoppa, men hon sa 'nej'\,''. När du beskriver situationen, tänk på vad du skulle ha sett om du hade filmat den. Det är viktigt att du i det här steget håller dig till en beskrivning av hur det var, utan att lägga med tolkningar eller värderingar. Skriv till exempel inte ''min vän var oförskämd mot mig'' eftersom det är ett antagande och en tolkning som vi ännu inte har några bevis för.

I nästa steg \textbf{identifierar du känslan} i situationen.

{\raggedright

\begin{itemize}

\item Vilken känsla eller vilka känslor hade jag?

\item Hur starkt upplevde jag den eller dem? Ange intensiteten på en skala från 0~till~100.

\end{itemize}

} % end raggedright

I exemplet nedan känner sig personen sårad och irriterad. Skattningarna av intensitet görs individuellt för varje känsla; skattningarna behöver inte bli hundra om du slår ihop dem.

Därefter \textbf{beskriver du ditt beteende} och de fysiologiska reaktionerna du kände i kroppen. Fråga dig själv:

\begin{itemize}

\item Vad gjorde jag?

\item Vad kände jag i kroppen?

\end{itemize}

I det här exemplet ignorerade personen ett telefonsamtal från en vän och kände sig spänd och mådde dåligt så fort hon tänkte på situationen.

Därefter \textbf{identifierar du ditt tänkande} i situationen. Tänkandet kan ta form av  antaganden, tolkningar, föreställningar, värderingar och så vidare. Ibland kan det till och med vara bilder, istället för ord. Fråga dig själv:

\begin{itemize}

\item Vad tänkte jag?

\item Vad for genom mitt huvud?

\end{itemize}

I exemplet som följer var personens tankar de här:

\begin{itemize}

\item ''Jag sade ja till att gå och shoppa med henne en gång när jag inte ville göra det.''

\item ''Därför borde hon ha sagt ja till mig den här gången.''

\item ''Att säga nej är oförskämt och själviskt, och det visar att hon inte bryr sig.''

\item ''Hon kanske inte tycker om mig längre.''

\end{itemize}

Sedan anger du \textbf{hur mycket du tror på tankarna}, för var och en av dem. En skattning på ''0'' betyder att du inte tror på tanken alls, och en skattning på ''100'' betyder att du tror på den till 100\%. När du har gjort klart den här delen av tankedagboken hoppar du vidare till nästa steg. I det testar du var och en av tankarna med de här frågorna:

\begin{itemize}

\item Är det är ett passivt, självhävdande eller aggressivt sätt att tänka?

\item Är det är ett passivt, självhävdande eller aggressivt sätt att reagera och handla?

\item Vilka bevis kan jag hitta som talar för att tanken stämmer?

\item Vilka bevis kan jag hitta som talar för att tanken inte stämmer?

\item Kränker jag mina rättigheter eller den andra personens rättigheter?

\item Finns det något annat sätt som jag kan se situationen på?

\item Vilka andra tolkningar av vad som hände kan jag göra?

\end{itemize}

Anledningen till att du får ställa dig de här frågorna är för att hjälpa dig att \textbf{formulera ett mer balanserat och konstruktivt självhävdande sätt att tänka och handla}. Det hjälper dig att ställa den här frågan till dig själv:

\begin{itemize}

\item Vad skulle kunna vara ett mer konstruktivt sätt att hävda mig själv i tanke och beteende?

\end{itemize}

Det sista steget handlar om att \textbf{återigen skatta intensiteten i den ursprungliga känslan och hur mycket du tror på din ursprungliga tanke}. Om du har fyllt i hela din tankedagbok är det möjligt att du kommer att konstatera en minskad grad av upplevd känsla, och en minskning i hur mycket du tror på din ursprungliga, negativa tanke. 

Om du fortsätter att öva på att tänka på det här sättet kan du märka att du kommer att börja tänka och hävda dig själv på ett mer konstruktivt sätt.

\subsection{Exempel på en tankedagbok}\label{tankedagbok}

På följande sidor får du se ett exempel på en tankedagbok. I exemplet har personen frågat sin vän om hon vill följa med att shoppa, men fått ett 'nej' som blev svårt att hantera och som gav upphov till passiv kommunikation.

%%%%%%%%%%%%%%%%%%%%%%%%%%%%%%%%%%%%%%%%%%%%%%%%%%%
% Some new macros and longtable settings
%%%%%%%%%%%%%%%%%%%%%%%%%%%%%%%%%%%%%%%%%%%%%%%%%%%
\newcolumntype{R}{>{\raggedleft\arraybackslash}m{.3\textwidth}}
\newcolumntype{L}{>{\raggedright\arraybackslash}m{.7\textwidth}}
\newcolumntype{M}{>{\raggedright\arraybackslash}m{.4\textwidth}}
\newcolumntype{C}{>{\centering\arraybackslash}m{.3\textwidth}}
\newcolumntype{D}{>{\centering\arraybackslash}m{.15\textwidth}}
%https://tex.stackexchange.com/questions/153771/how-to-set-the-width-of-column-in-longtable-method
\newcolumntype{A}{>{\raggedleft\arraybackslash}X} % flush-left, while allowing hyphenation
\newcolumntype{B}{>{\raggedright\arraybackslash}X} % flush-left, while allowing hyphenation
\setlength\tabcolsep{0pt}
%\setlength\tabcolsep{.5\baselineskip}
\newcommand\xxx[1]{%
  \begin{minipage}[c][4.25\baselineskip][c]{.3\textwidth}%
  \raggedleft\strut #1\strut%
  \end{minipage}%
}
\newcommand\xy[2]{%
  \begin{minipage}[c][4.25\baselineskip][c]{#1}%
  \raggedleft\strut #2\strut%
  \end{minipage}%
}
\newcommand\xyh[2]{%
  \begin{minipage}[c][3.25\baselineskip][c]{#1}%
  \raggedleft\strut #2\strut%
  \end{minipage}%
}
% \newcommand\xyz[2]{%
%   \begin{minipage}[c][4.75\baselineskip][c]{#1}%
%   \raggedleft\strut #2%
%   \end{minipage}%
% }
\newfontfamily{\handwriting}[Scale=MatchUppercase,WordSpace={.75}]{Tillana}
\newcommand\yyy[1]{%
  \hspace{\baselineskip}\begin{minipage}[c][4.25\baselineskip][c]{.7\textwidth-\baselineskip}%
  \handwriting\raggedright\strut\Large #1\strut%
  \end{minipage}%
}
\newcommand\yz[2]{%
  \hspace{\baselineskip}\begin{minipage}[c][4.25\baselineskip][c]{#1-\baselineskip}%
  \handwriting\raggedright\strut\Large #2\strut%
  \end{minipage}%
}
\newcommand\yzh[2]{%
  \hspace{\baselineskip}\begin{minipage}[c][3.25\baselineskip][c]{#1-\baselineskip}%
  \handwriting\raggedright\strut\Large #2\strut%
  \end{minipage}%
}
%%%%%%%%%%%%%%%%%%%%%%%%%%%%%%%%%%%%%%%%%%%%%%%%%%%

\newpage

\subsubsection{Del 1 -- Förstå din reaktion}

{\linespread{.95}
\begin{longtable}[]{@{}RL@{}}
\toprule
\xxx{Situation} & \yyy{Frågade Lisa om hon ville följa med och shoppa, hon sa 'nej'}\\
\midrule
\xxx{Vilka var mina känslor? Hur starka var de? \\ Skatta från 0 till 100}& \yyy{Sårad -- 70\\Arg -- 80}\\
\midrule
\xxx{Vilka fysiska responser märkte jag i kroppen?}& \yyy{Spänd, trångt över bröstet, bet ihop käkarna. Kände mig sjuk när jag\\ tänkte på henne.}\\
\midrule
\xxx{Vad gjorde jag?} & \yyy{Grät, svarade inte när hon ringde} \\
\midrule
\xxx{Var det här passivt, självhävdande eller aggressivt beteende?}  & \yyy{Passivt aggressivt} \\
\midrule
\xxx{Vilka tankar for genom mitt huvud?}  & \yyy{Jag sa 'ja' förut fast jag inte ville shoppa. Hon borde ha sagt 'ja' nu. Hon är självisk. Hon tycker inte om mig.} \\
\midrule
\xxx{Vilken är den\\ ''hetaste'' tanken?}  & \yyy{Hon är självisk som säger 'nej'} \\
\midrule
\xxx{Hur mycket tror jag på den heta tanken?}  & \yyy{80} \\
\midrule
\xxx{Är det här en passiv, självhävdande eller aggressiv tanke?}  & \yyy{Passiv för jag tänkte jag borde göra något jag inte ville. Aggressiv för att Lisa måste göra som jag ville.} \\
\bottomrule
\end{longtable}
}

\newpage

\subsubsection{Del 2 -- Balansera din negativa tanke}

{\linespread{.95}
\begin{longtable}[]{@{}RL@{}}
\toprule
\xxx{Vilka bevis har jag för att tanken stämmer?} & \yyy{Inga} \\
\midrule
\xxx{Vilka bevis har jag för att tanken inte stämmer?} & \yyy{Vi har gjort massor av saker så länge vi känt varandra} \\
\midrule
\xxx{Kränker jag mina rättigheter eller någon annans rättigheter?\\ I sådana fall, hur då?} & \yyy{Ja, jag kränkte mina när jag sa 'ja' mot min vilja. Jag kränker hennes rätt att säga 'nej' när jag kräver att hon följer med.} \\
\midrule
\xxx{På vilka andra sätt kan jag tolka situationen?} & \yyy{\large\strut Hon kanske var trött. Eller inte ville. Eller hade andra planer. Jag försökte ''tankeläsa'' henne. Jag brukar också säga 'nej' ibland, fast jag gillar henne mycket.} \\
\midrule
\xxx{Vad skulle vara ett mer konstruktivt själv-\\ hävdande sätt att tänka?} & \yyy{Hon har rätt att säga 'nej'. Det betyder inte att hon är självisk. Det säger inget om vad hon tycker om mig.} \\
\midrule
\xxx{Vad skulle vara ett mer konstruktivt själv-\\ hävdande sätt att agera?} & \yyy{Jag vill fråga om vi kan ses en annan gång och göra nåt som båda vill!} \\
\midrule
\xxx{Skatta initiala känslan igen. Skatta hur mycket du tror på första tanken igen (båda 0--100)} & \yyy{Sårad -- 20\\Arg -- 10\\Tror på tanken -- 10} \\
\bottomrule
\end{longtable}
}

\newpage

Testa att använda en tankedagbok nästa gång du märker att du känner dig sårad, arg eller upprörd, efter en interaktion med en annan person. Det kan hända att du tänkte och/eller reagerade på ett sätt som inte var självhävdande. Fortsätt att använda tankedagböcker för liknande situationer, tills det blir en naturlig reaktion och du märker att du börjar göra det spontant. Då kommer du märka att du kan fånga in ditt eget negativa (självförminskande eller aggressiva) tänkande och ge dig tid att balansera det, innan du agerar på det. Det kan ta ganska lång tid innan du kommer fram till det här stadiet. Det är därför du börjar med att öva att utmana ditt sätt att tänka med hjälp av en tankedagbok.

\subsection{Din tankedagbok}

Nu är det din tur. Fyll i en egen tankedagbok på sidan~\pageref{form:tankedagbok}. Välj en situation i vilken du blev sårad, arg eller upprörd, och som du hade svårt att hävda dig i.

\newpage

\subsubsection{Del 1 -- Förstå din reaktion}

{\linespread{.95}\label{form:tankedagbok}
\begin{longtable}[]{@{}RL@{}}
\toprule
\xxx{Situation} &\\
\midrule
\xxx{Vilka var mina känslor? Hur starka var de? \\ Skatta från 0 till 100}&\\
\midrule
\xxx{Vilka fysiska responser märkte jag i kroppen?}&\\
\midrule
\xxx{Vad gjorde jag?}&\\
\midrule
\xxx{Var det här passivt, självhävdande eller aggressivt beteende?}&\\
\midrule
\xxx{Vilka tankar for genom mitt huvud?}&\\
\midrule
\xxx{Vilken är den \\''hetaste'' tanken?}&\\
\midrule
\xxx{Hur mycket tror jag på den heta tanken?}&\\
\midrule
\xxx{Är det här en passiv, självhävdande eller aggressiv tanke?}&\\
\bottomrule
\end{longtable}
}

\newpage

\subsubsection{Del 2 -- Balansera din negativa tanke}

{\linespread{.95}
\begin{longtable}[]{@{}RL@{}}
\toprule
\xxx{Vilka bevis har jag för att tanken stämmer?} &\\
\midrule
\xxx{Vilka bevis har jag för att tanken inte stämmer?}&\\
\midrule
\xxx{Kränker jag mina rättigheter eller någon annans rättigheter?\\ I sådana fall, hur då?}&\\
\midrule
\xxx{På vilka andra sätt kan jag tolka situationen?}&\\
\midrule
\xxx{Vad skulle vara ett mer konstruktivt själv-\\ hävdande sätt att tänka?}&\\
\midrule
\xxx{Vad skulle vara ett mer konstruktivt själv-\\ hävdande sätt att agera?}&\\
\midrule
\xxx{Skatta initiala känslan igen. Skatta hur mycket du tror på första tanken igen (båda 0--100)}&\\
\bottomrule
\end{longtable}
}

\index{tankedagbok|)}
\index{antagande!hindrande!balansera|)}
\index{balansera antagande|)}

\newpage

\subsection{Beteendeexperiment}

\index{beteendeexperiment|(}
\index{antagande!hindrande!balansera|(}
\index{balansera antagande|(}

En tankedagbok kan hjälpa oss att utmana negativa automatiska tankar om en situation. Det gäller särskilt när vi upprepade gånger skriver ned de negativa tankarna och aktivt övar på att ge plats för alternativa tankar. Men ibland räcker det inte med att skriva ned negativa tankar och nya, balanserande tankar. Det kan kännas som att de balanserade tankarna visserligen är logiska, men att inget i praktiken ändras. Ett annat problem med en tankedagbok uppstår när det inte finns några tydliga bevis för eller emot den negativa tanken i en viss situation. Om vi till exempel har antagandet att 
%''Om vi inte alltid gör vad andra vill, så kommer de inte att tycka om oss längre''
''Om vi slutar göra vad andra vill, så kommer de inte att tycka om oss längre'', är risken stor att vi aldrig utsätter oss för situationer där vi kan pröva om antagandet stämmer. Eftersom vi sannolikt inte vill utsätta oss för risken att få rätt i vår negativa förutsägelse, räcker det inte med att använda enbart en tankedagbok. Det vi saknar är övertygande bevis för och emot den negativa tanken.

I en sådan situation är ett beteendeexperiment mycket användbart. Precis som med tankedagboken handlar det om att balansera automatiskt tänkande, men vi gör det genom att testa hållbarheten i tankeinneållet, istället för att bara tänka annorlunda om antagandet.

Någon kanske till exempel har tanken ''Om jag hävdar mig själv kommer den andra personen att bli upprörd, och vår relation kommer att bli förstörd''. För personen som tänker så har antagandet inneburit att han eller hon alltid har gått med på vad andra föreslagit, även om det har gått helt emot vad personen egentligen har velat göra. Antagandet är så starkt att personen inte ens kan tänka sig att hävda sina egna åsikter. Personen kanske har fyllt i sin tankedagbok och upptäckt att det inte finns något övertygande bevis för att antagandet stämmer, men å andra sidan har hon eller han inte heller upptäckt några övertygande bevis för att det inte stämmer. Personen har kört fast.

Ett sätt att komma loss är att göra ett experiment som går ut på att pröva giltigheten i förutsägelsen att den andra personen kommer att bli upprörd och att relationen kommer att gå i kras. Att göra experimentet kan vara utmanande och väcka en hel del ångest. När du bestämt dig för att göra ett experiment är det därför viktigt att du planerar det noggrant, så att resultaten blir användbara för dig. Du kan också behöva börja med ett lite enklare experiment och gradvis gå vidare med allt mer utmanande situationer.

Här beskriver vi ett experiment som prövar antagandet ''Om jag hävdar mig själv kommer den andra personen att bli upprörd, och vår relation kommer att bli förstörd''.

\begin{enumerate}

\item 
Det första steget är att identifiera förutsägelsen som du har om situationen och om hur du kan avgöra om förutsägelsen stämmer. Det här är ett viktigt steg. Om du inte tydligt kan avgöra om det antagandet säger faktiskt har inträffat, så är risken att du själv flyttar målstolparna under eller efter experimentet, och inte upptäcker skillnaden.

\item 
Du behöver också identifiera de ej självhävdande beteenden som du brukar ta till, så att du kan vara säker på att du utför experimentet utan att göra de sakerna.

\item 
Därefter skriver du ned ett antal mer positiva och hjälpsamma förutsägelser om vad som kan hända. Dessa kan komma från en tidigare ifylld tankedagbok, eller så kan du identifiera dem utan en tankedagbok.

\item 
Därefter planerar du själva experimentet. I det ingår att tydligt bestämma vilka steg som ska ingå i genomförandet. Gör det tydligt för dig själv när, var och hur experimentet ska gå till. Här identifierar du också konstruktiva självhävdande beteenden som du kommer att använda dig av i experimentet.

\item 
Sedan gör du experimentet och utvärderar resultaten. Fråga dig själv:

\begin{itemize}

\item Vad hände?

\item Fick du stöd för dina ursprungliga negativa förutsägelser?

\item Vad lärde du dig av experimentet?

\end{itemize}

\end{enumerate}

\subsection{Exempel på ett beteendeexperiment}

Titta på exemplet på sidan~\pageref{form:beteendeexperiment-1} för att se hur man kan lägga upp ett experiment. 

\newpage

\subsubsection{Steg 1 -- Identifiera din förutsägelse}

{\linespread{.95}\label{form:beteendeexperiment-1}
\begin{longtable}[]{@{}RL@{}}
\toprule
\xxx{Situation} & \yyy{Om jag egentligen är upptagen ska jag säga 'nej' nästan gång Lisa frågar om jag vill följa med och handla kläder.} \\
\midrule
\xxx{Förutsägelse} & \yyy{Hon kommer bli upprörd och arg och vill inte längre vara min vän.} \\
\midrule
\xxx{Hur mycket tror jag att det ska hända (0--100)} & \yyy{70} \\
\midrule
\xxx{Hur vet jag om det\\ har hänt?} & \yyy{Hon kommer lägga på luren, inte ringa upp igen om jag ringer, inte prata med mig.} \\
\bottomrule
\end{longtable}
}
\subsubsection{Steg 2 -- Identifiera hindrande beteenden}

{\linespread{.95}
\begin{longtable}[]{@{}RL@{}}
\toprule
\xxx{Sådant jag brukar göra för att klara mig\\ (t. ex. undvikande, flykt, säkerhetsbeteenden)} & \yyy{Hitta på ursäkter, låtsas vara sjuk, undvika henne tills det är för sent.} \\
\bottomrule
\end{longtable}
}

\subsubsection{Steg 3 -- Identifiera en mer realistisk förutsägelse}

{\linespread{.95}
\begin{longtable}[]{@{}RL@{}}
\toprule
\xxx{Påminn mig själv om en mer realistisk förutsäg-\\else (kan komma från tankedagbok)} & \yyy{\large Hon sa 'nej' till mig förut och det var okej med mig. Hon kommer kanske bli besviken men om hon är en god vän kommer hon inte bli arg och vi kommer vara vänner.} \\
\bottomrule
\end{longtable}
}

\newpage

\subsubsection{Steg 4 -- Identifiera mina konstruktiva beteenden}

{\linespread{.95}
\begin{longtable}[]{@{}RL@{}}
\toprule
\xxx{Vad kommer jag göra annorlunda, för att testa de två förutsägelserna?} & \yyy{\large Istället för att undvika henne ska jag ringa upp direkt och förklara att jag har ett möte jag måste förbereda och inte kan gå och shoppa. Jag bokar in en annan tid med henne istället.} \\
\bottomrule
\end{longtable}
}

\subsubsection{Steg 5 -- Utför och utvärdera experimentet}

{\linespread{.95}
\begin{longtable}[]{@{}RL@{}}
\toprule
\xxx{Vad hände faktiskt?} & \yyy{Hon sa att det var lugnt och faktiskt passade henne också. Vi ska ta en fika på onsdag nästa vecka.} \\
\midrule
\xxx{Hur väl stämde min första förutsägelse (0--100)} & \yyy{0} \\
\midrule
\xxx{Vilken förutsägelse\\ fick stöd?} & \yyy{Den mer realistiska} \\
\midrule
\xxx{Hur var det att bete\\ sig annorlunda?} & \yyy{Läskigt, svårt i början, sen lättare} \\
\midrule
\xxx{Vad lärde jag mig\\ av experimentet?} & \yyy{\large Jag brukar förvänta mig det värsta och då är jag ofta passiv. Det kändes bra att vara ärlig och rak. Om jag hävdar mig själv blir oftast inte andra arga, om de ändå blir det är det deras problem att jobba med.} \\
\bottomrule
\end{longtable}
}

\subsection{Mitt beteendeexperiment}

På sidan~\pageref{form:beteendeexperiment-1-egen} finns ett tomt experiment-formulär för dig att arbeta med. Gå genom det med en av dina hindrande förutsägelser, och utforma ett eget beteendeexperiment!

\newpage

\subsubsection{Steg 1 -- Identifiera din förutsägelse}

{\linespread{.95}\label{form:beteendeexperiment-1-egen}
\begin{longtable}[]{@{}RL@{}}
\toprule
\xxx{Situation} & \yyy{} \\
\midrule
\xxx{Förutsägelse} & \yyy{} \\
\midrule
\xxx{Hur mycket tror jag att det ska hända (0--100)} & \yyy{} \\
\midrule
\xxx{Hur vet jag om det\\ har hänt?} & \yyy{} \\
\bottomrule
\end{longtable}
}

\subsubsection{Steg 2 -- Identifiera hindrande beteenden}

{\linespread{.95}
\begin{longtable}[]{@{}RL@{}}
\toprule
\xxx{Sådant jag brukar göra för att klara mig\\ (t. ex. undvikande, flykt, säkerhetsbeteenden)} & \yyy{} \\
\bottomrule
\end{longtable}
}

\subsubsection{Steg 3 -- Identifiera en mer realistisk förutsägelse}

{\linespread{.95}
\begin{longtable}[]{@{}RL@{}}
\toprule
\xxx{Påminn mig själv om en mer realistisk förutsäg-\\else (kan komma från tankedagbok)} & \yyy{} \\
\bottomrule
\end{longtable}
}

\newpage

\subsubsection{Steg 4 -- Identifiera mina konstruktiva beteenden}

{\linespread{.95}
\begin{longtable}[]{@{}RL@{}}
\toprule
\xxx{Vad kommer jag göra annorlunda, för att testa de två förutsägelserna?} & \yyy{} \\
\bottomrule
\end{longtable}
}

\subsubsection{Steg 5 -- Utför och utvärdera experimentet}

{\linespread{.95}
\begin{longtable}[]{@{}RL@{}}
\toprule
\xxx{Vad hände faktiskt?} & \yyy{} \\
\midrule
\xxx{Hur väl stämde min första förutsägelse (0--100)} & \yyy{} \\
\midrule
\xxx{Vilken förutsägelse\\ fick stöd?} & \yyy{} \\
\midrule
\xxx{Hur var det att bete\\ sig annorlunda?} & \yyy{} \\
\midrule
\xxx{Vad lärde jag mig\\ av experimentet} & \yyy{} \\
\bottomrule
\end{longtable}
}

\index{beteendeexperiment|)}
\index{antagande!hindrande!balansera|)}
\index{balansera antagande|)}

\newpage

\section{Sammanfattning}

\begin{itemize}

\item Vårt sätt att tänka kan ibland hindra oss från att hävda oss själva konstruktivt.

\item Att hävda sig själv är att hävda sina rättigheter, men med dem kommer också skyldigheter. Vi har ett ansvar mot oss själva och personer som vi interagerar med.

\item Du kan ändra ditt tänkande om det hindrar dig från konstruktiv självhävdelse. Ett sätt är att använda kognitiv beteendeterapi (KBT). Med KBT använder du hjälpmedel som tankedagböcker och utför beteendeexperiment.

\end{itemize}

\nextup{I nästa kapitel går vi genom grundläggande tekniker för konstruktiv självhävdelse. De är användbara i många situationer.}

% KORR AV UTSKRIFT HIT 190422

\chapter{Beteenden för konstruktiv självhävdelse}\label{chap4}

\section{Tekniker för konstruktiv självhävdelse}

I det här kapitlet introducerar vi några allmänna tekniker för konstruktiv självhävdelse. Dessa tekniker kan användas i en stor mängd situationer. I kapitel~\ref{chap7} finns ytterligare självhävdelse-tekniker för att handskas med kritik.

När du övar på de här teknikerna kan det vara bra att börja i en neutral situation. Med det menar vi en situation där dina känslor inte är för starka. När du efter hand blir mer och mer förtrogen med och skicklig på att utföra teknikerna, kan du börja använda dem i svårare och mer känslosamma situationer. Precis som när du lär dig andra färdigheter är det viktigt att komma ihåg att det inte alltid går som du tänkt eller planerat. Det är viktigt att du inte klandrar dig själv för det, utan tittar på vad det var som gick fel och vad som du kan pröva att ändra nästa gång. Och att du sedan testar tekniken igen! Med tiden kommer du märka att det blir lättare och lättare.

%%%%%%%%%%%%%%%%%%%%%%%%%%%%%%%%%%%%
% https://tex.stackexchange.com/questions/46434/how-to-highlight-text-formulas-with-tikz
\newcommand*{\xwarning}[1]{%
  \vspace{1.75\baselineskip}
  \tikz[baseline=(X.base)] \node[rectangle, fill=accent, text=white, inner sep=.25\baselineskip] (X) {\sffamily\bfseries\fontsize{1\baselineskip}{1\baselineskip}\selectfont #1};%
  \vspace{.5\baselineskip}
}
%%%%%%%%%%%%%%%%%%%%%%%%%%%%%%%%%%%%

\xwarning{Observera! Kom ihåg den icke-verbala kommunikationen!}\par\nobreak

Med alla dessa tekniker är det lika viktigt att tänka på den icke-verbala kommunikationen, som den verbala. Det kan vara lätt att tro att man hävdar sig själv konstruktivt bara för att man använder en viss teknik för självhävdande; det är dock lika möjligt att använda dessa tekniker på ett aggressivt eller passivt sätt, om du inte är noga med den icke-verbala delen av kommunikationen.

Se till att du hävdar dig själv konstruktivt genom att hålla ett lågt röstläge, normal röststyrka, jämn taltakt och god ögonkontakt. Försök slappna av i kroppen, så att du håller den fysiska spänning låg. Gå tillbaka till kapitel~\ref{chap2}, \textit{\nameref{chap2}}, om du behöver repetera olika icke-verbala sätt som du kan hävda dig själv konstruktivt på.

\section{Grundläggande självhävdelse}

Grundläggande konstruktiv självhävdelse är när vi säger något på ett sätt som förmedlar våra behov, önskningar, förväntningar, åsikter eller känslor. Den här typen av konstruktiv självhävdelse kan användas varje dag för kommunicera behov. Vanligtvis inleds grundläggande konstruktivt självhävdande med ''Jag \ldots{}''. Exempel på jag-budskap är:

\begin{itemize}

\item ''Jag behöver lämna jobbet klockan 5.''

\item ''Jag är glad över sättet vi kunde lösa problemet på.''

\end{itemize}

Du kan också använda tekniken när du vill berömma eller komplimera någon, när du vill ge någon information eller när du vill ta upp något viktigt för första gången. Till exempel:

\begin{itemize}

\item ''Jag har inte tänkt på det förut, jag vill ge mig lite tid att tänka på din idé.''

\item ''Jag tycker att din presentation verkligen var bra.''

\item ''Priset kommer att bli 20\,000 kronor.''

\item ''Jag blir glad när du hjälper mig.''

\end{itemize}

Det är viktigt att komma ihåg att vara specifik när du säger något på det här sättet. Bestäm dig för vad du vill eller vad du känner, och säg det direkt och specifikt. Undvik onödig utfyllnad och håll din kommunikation enkel och koncis. Den här färdigheten hjälper dig att vara tydlig med exakt det du vill kommunicera.

I grundläggande konstruktiv självhävdelse ingår också \textbf{självutlämnande} (eller själv-avslöjande) tekniker. Med det menas att du säger vad du känner med en enkel mening, till exempel så här:

\begin{itemize}

\item ''Jag känner mig nervös.''

\item ''Jag känner skuld.''

\item ''Jag känner mig arg.''

\end{itemize}

Den omedelbara effekten efter att du använt tekniken är minskad ångest eller minskat obehag, vilket gör dig mer avspänd. Det ger dig en chans att återta kommandot över dig själv och dina känslor. Genom att använda jag-budskap för att uttrycka dina känslor på det här sättet visar du också för dem omkring dig att du tar ansvar för dina känslor.

\section{Empatisk självhävdelse}

Empatisk självhävdelse betyder att vi hävdar oss själva samtidigt som vi försöker förstå en annan persons känslor, behov eller önskemål. Den här typen av konstruktiv självhävdelse kombinerar alltså att vi säger vad vi själva behöver och vill med ett empatiskt erkännande av den andra personen.

Empatiskt självhävdande kan användas när du är inblandad i en situation som inte stämmer överens med dina behov. Samtidigt som du säger vad du tänker eller känner visar du att du är lyhörd för hur den andra personen har det.

Exempel på empatisk självhävdelse:

\begin{itemize}

\item ''Jag förstår att du inte gillar det nya arbetssättet. Tills vi kan ändra det igen vill jag ändå att du gör på det nya sättet.''

\item ''Jag vet att du är upptagen nu, Johan, men jag skulle vilja be dig om en sak.''

\item ''Jag har full förståelse för att det är svårt att säga exakt vad priset kommer att bli, jag skulle ändå behöva en uppskattning, mellan tummen och pekfingret.''

\end{itemize}

Empatisk självhävdelse kan hjälpa dig att hålla igen från att överreagera på ett aggressivt sätt. Med empatisk självhävdelse ger du dig själv lite betänketid, för att bättre förstå i vilken sits den andra personen befinner sig.

Det är möjligt att överanvända vissa empatiskt självhävdande fraser. Då kan det börja låta som att du inte är uppriktig. I exemplet ''Jo, jag förstår dina känslor, men \ldots{}'', så vattnas den empatiska delen ''jag förstår dina känslor'' ut av det följande ordet ''men''. Frasen ser ut som konstruktiv självhävdelse men blir istället ett uttryck för aggressivitet.

\section{Konsekvensbaserad självhävdelse}

Det här är den starkaste sortens självhävdelse och är ett beteende som används som sista utväg. Det används vanligtvis i en situation när någon inte har respekterat andras rättigheter och du vill påtala detta utan att själv bli aggressiv. I en jobb-situation kan det användas när överenskomna arbetssätt eller riktlinjer inte följs. Konsekvensbaserad självhävdelse går ut på att informera personen om konsekvenserna om han eller hon inte ändrar beteende. Det kan väldigt lätt uppfattas som hotande och därför aggressivt. Använd bara det här sättet att hävda dig själv när du har sanktioner (konsekvenser) som du kan ta till, och bara om du också är beredd att göra det.

Eftersom den här typen av självhävdelse lätt kan uppfattas som aggressivitet behöver du vara mycket försiktig med de icke-verbala signaler som du sänder. Håll rösten låg och i en jämn nivå och volym, ha god ögonkontakt och försök att hålla kroppen och ansiktet avslappnade.

Exempel på konsekvensbaserad självhävdelse:

\begin{itemize}

\item ''Om du fortsätter att undanhålla mig information har jag inget annat val än att ta detta till produktionschefen. Jag skulle helst slippa det.''

\item ''Johan, jag är inte beredd på att låta mina medarbetare samarbeta med dina i projektet såvida du inte ger dem tillgång till samma lokaler och utrustning som dina medarbetare har.''

\item ''Om det här händer igen har jag inget annat val än att dra igång ett formellt disciplinärende. Jag skulle föredra att inte göra det.''

\end{itemize}

\section{Diskrepansbaserad självhävdelse}

Diskrepansbaserad självhävdelse går ut på att peka på skillnaden mellan vad som tidigare överenskommits och vad som faktiskt händer. Det är användbart för att bringa klarhet i om det finns någon motsättning i åsikt eller om det föreligger ett missförstånd, när ord och beteende inte stämmer överens för en person.

Två exempel på självhävdande i situationer med skillnad i ord och handling:

\begin{itemize}

\item ''Som jag förstod det kom vi överens om att projekt A hade högsta prioritet. Nu ber du om mer tid för projekt B. Jag skulle vilja få klarhet i vilket projekt som just nu är viktigast.''

\item ''Paul, å ena sidan säger du att du vill förbättra samarbetet mellan avdelningarna, men å andra sidan säger du saker om oss som gör det svårt för oss att samarbeta. Jag håller med om att vi kan förbättra situationen, så jag skulle vilja prata om det.''

\end{itemize}

\section{Självhävdelse vid negativa känslor}

När du är sårad i en relation, eller känner starka negativa känslor gentemot en annan person -- som ilska eller förakt -- kan du använda dig av följande självhävdande teknik: På ett kontrollerat och lugnt sätt uppmärksammar du personen om vilka oönskade effekter den andras beteende har på dig. Det ger dig möjlighet att hantera de jobbiga känslorna, utan att tappa humöret på ett okontrollerat sätt. Och det ger den andra personen möjlighet att förstå på vilket sätt hans eller hennes handlande påverkar dig.

\subsection{Självhävdelse vid starka, negativa känslor}

\vbox{%
{\linespread{.95}
\begin{longtable}[]{@{}p{.4\textwidth}m{.6\textwidth}@{}}
\toprule
\xyh{.4\textwidth}{\sffamily\bfseries\normalsize\centering Steg\strut} & %
\yzh{.6\textwidth}{\sffamily\bfseries\normalsize\centering Exempel\strut} \\ %
\midrule
\xy{.4\textwidth}{1. Beskriv den andra personens beteende objektivt. Gör det utan att tolka eller döma.\strut} & 
\yz{.6\textwidth}{\strut När du väntar så här länge med att\\ sätta igång med din rapport \ldots{}\strut} \\
\midrule
\xy{.4\textwidth}{2. Beskriv vilken inverkan den andra personens beteende får på dig. Var specifik och tydlig. Övergeneralisera inte.\strut} & 
\yz{.6\textwidth}{\strut\ldots{} betyder det att jag behöver\\ jobba över i helgen.\strut} \\
\midrule
\xy{.4\textwidth}{3. Beskriv dina känslor.\strut} & 
\yz{.6\textwidth}{\strut Jag känner mig irriterad\\ över det här \ldots{}\strut} \\
\midrule
\xy{.4\textwidth}{4. Säg vilket framtida beteende du önskar.\strut} & 
\yz{.6\textwidth}{\strut\ldots{} så i framtiden vill jag ha rapporten senast fredag lunch.\strut} \\
\bottomrule
\end{longtable}
}
}

Här är två exempel på självhävdelse vid negativa känslor:

\begin{itemize}

\item ''När du kommer hem sent, utan att säga till om det innan, oroar jag mig för att något är fel och då blir jag arg. Jag skulle verkligen uppskatta om du kunde slå en signal och säga till innan.''

\item ''Varje gång du avbryter mig när jag jobbar med balansräkningen måste jag börja om igen. Jag känner mig irriterad över det, så jag skulle föredra att du väntade tills jag är klar.''

\end{itemize}

\section{Repig grammofonskiva-teknik}

Små barn är experter på den här tekniken, med vilken du upprepar din ståndpunkt som en repig gammal grammofonskiva av vinyl, som hakat upp sig. Den här färdigheten går ut på att först komma på vad du ska säga, och sedan repetera det så många gånger som det behövs, på ett lugnt och avslappnat sätt. Den här färdigheten är användbar i de flesta situationer. Det är en bra teknik att använda när du har att göra med smarta, vältaliga personer, eftersom det enda du måste göra är att hålla dig till det du vill säga. Tekniken hjälper dig att behålla lugnet eftersom du inte behöver tänka på vad du ska säga för att inte falla i irrelevanta logiska fällor; du behöver bara hålla rak kurs mot ditt mål. Tekniken är särskilt bra i situationer när du vill säga 'nej' (vi går in på det här mer i detalj i kapitel~\ref{chap6}, \textit{\nameref{chap6}}).

\subsection{Exempel på repig grammofonskiva-tekniken}

\begin{description}

\item[Karin] -- Kan jag låna 200 kronor av dig?

\item[David] -- Jag kan inte låna ut några pengar, jag är luspank.

\item[Karin] -- Jag betalar tillbaka så fort jag kan. Jag är desperat! Kom igen, var en kompis!

\item[David] -- Jag kan inte låna ut några pengar.

\item[Karin] -- Om det var jag skulle jag göra det. Du kommer inte sakna 200 kronor.

\item[David] -- Jag är din kompis men jag kan inte låna dig pengar, jag är pank.

\end{description}

Repig grammofonskiva-tekniken kan kombineras med övriga tekniker för konstruktiv självhävdelse som du har lärt dig. Börja alltid med den mildaste hållningen, och bygg på utifrån vad du bedömer behövs för situationen. Undvik att direkt gå på den starkaste konsekvensen, eftersom det kan uppfattas som ett hot och som aggressivt beteende, snarare än konstruktiv självhävdelse.

Det följande exemplet av repig grammofonskiva-tekniken använder alla nivåerna av självhävdelse, från grundläggande, via empatisk till konsekvensbaserad självhävdelse.

\subsection{Grundläggande}

\begin{description}

\item[Kunden] -- Jag köpte den här klockan igår. Vredet för att ställa om tiden fungerar inte som det ska, så jag vill byta den.

Här kan expediten gå med på ditt önskemål eller kanske säga: 

\item[Expediten] -- Du skulle ha kontrollerat klockan innan du lämnade affären.

\end{description}

\subsection{Empatisk}

\begin{description}

\item[Kunden] -- Jag förstår att det skulle ha gjort saken lättare för dig, men jag vill ända byta ut den.

Här kan expediten gå med på ditt önskemål eller säga något i stil med: 

\item[Expediten] -- Jag har inte mandat att byta ut varor.

\item[Kunden] -- Jag vill ända få den utbytt.

Efter en stunds ordväxling kan nivån höjas till konsekvensen.

\end{description}

\subsection{Konsekvens}

\begin{description}

\item[Kunden] -- Jag vill få klockan utbytt. Om du inte är beredd att byta den kommer jag ta upp det med huvudkontoret. Jag skulle föredra att lösa problemet nu.

\end{description}

En nackdel med den här tekniken kan vara att personen du pratar med helt enkelt inte vill göra det du ber om. När han eller hon fortsätter att motstå din begäran riskerar den att förlora i kraft, varje gång du upp upprepar den. Om du upprepar din begäran för ofta kan det slå tillbaka på auktoriteten bakom dina ord. I sådana situationer är det nödvändigt att ha några konsekvenser till hands.

\section{Övning ger färdighet}

Färdighet i dessa tekniker kräver övning. Börja med grundläggande självhävdelse och öva på det under en vecka eller två, innan du går vidare med övriga tekniker. Välj och arbeta med en teknik i taget, och tillämpa den så fort det är lämpligt. En liten loggbok eller dagbok kan vara till god hjälp för att komma ihåg de tillfällen då du lyckats tillämpa teknikerna med framgång. Med hjälp av anteckningarna kan du också få en bild av hur ofta du använder teknikerna och se vilka som fungerar bäst för dig. Här är ett exempel på ett registreringsformulär. 

\subsection{Exempel på formulär för självhävdelseträning}

{\linespread{.95}
\begin{longtable}[]{@{}m{.15\textwidth}m{.2\textwidth}m{.3\textwidth}m{.35\textwidth}m{.25\textwidth}@{}}
\toprule
\xyh{.15\textwidth}{\sffamily\bfseries\normalsize\centering Dag \& tid\strut} & %
\yzh{.2\textwidth}{\sffamily\bfseries\normalsize\centering Använd teknik\strut} & %
\yzh{.3\textwidth}{\sffamily\bfseries\normalsize\centering Situation \&\\ tillämpning\strut} & %
\yzh{.35\textwidth}{\sffamily\bfseries\normalsize\centering Komma ihåg\\ till nästa gång\strut} \\
\midrule
\xy{.15\textwidth}{\raggedright\handwriting\Large Tisdag\\ kl 10\strut} & %
\yz{.2\textwidth}{\raggedright\handwriting\Large Grund-\\läggande\strut} & %
\yz{.3\textwidth}{\raggedright\handwriting\large På jobbet\\Gav Maria beröm för rapporten\strut} & %
\yz{.35\textwidth}{\raggedright\handwriting\large Rösten låg, tittade bort. Nästa gång: tala högre, ha ögonkontakt\strut} \\
\midrule
\xy{.15\textwidth}{\raggedright\handwriting\Large Ons 14:00\strut} & %
\yz{.2\textwidth}{\raggedright\handwriting\Large Diskrepans\strut} & %
\yz{.3\textwidth}{\raggedright\handwriting\large På jobbet: Chefen sa en sak, sen en annan, inom 5 min\strut} & %
\yz{.35\textwidth}{\raggedright\handwriting\large Blev nog för arg, lät irriterad. Behöver vara lugnare\strut} \\
\bottomrule
\end{longtable}
}

\newpage

\subsection{Registreringsformulär för självhävdelseträning}

{\linespread{.95}
\begin{longtable}[]{@{}m{.15\textwidth}m{.2\textwidth}m{.3\textwidth}m{.35\textwidth}m{.25\textwidth}@{}}
\toprule
\xyh{.15\textwidth}{\sffamily\bfseries\normalsize\centering Dag \& tid\strut} & %
\yzh{.2\textwidth}{\sffamily\bfseries\normalsize\centering Använd teknik\strut} & %
\yzh{.3\textwidth}{\sffamily\bfseries\normalsize\centering Situation \&\\ tillämpning\strut} & %
\yzh{.35\textwidth}{\sffamily\bfseries\normalsize\centering Komma ihåg\\ till nästa gång\strut} \\
\midrule
\xy{.15\textwidth}{} & %
\yz{.2\textwidth}{} & %
\yz{.3\textwidth}{} & %
\yz{.35\textwidth}{} \\
\midrule
\xy{.15\textwidth}{} & %
\yz{.2\textwidth}{} & %
\yz{.3\textwidth}{} & %
\yz{.35\textwidth}{} \\
\midrule
\xy{.15\textwidth}{} & %
\yz{.2\textwidth}{} & %
\yz{.3\textwidth}{} & %
\yz{.35\textwidth}{} \\
\midrule
\xy{.15\textwidth}{} & %
\yz{.2\textwidth}{} & %
\yz{.3\textwidth}{} & %
\yz{.35\textwidth}{} \\
\midrule
\xy{.15\textwidth}{} & %
\yz{.2\textwidth}{} & %
\yz{.3\textwidth}{} & %
\yz{.35\textwidth}{} \\
\midrule
\xy{.15\textwidth}{} & %
\yz{.2\textwidth}{} & %
\yz{.3\textwidth}{} & %
\yz{.35\textwidth}{} \\
\midrule
\xy{.15\textwidth}{} & %
\yz{.2\textwidth}{} & %
\yz{.3\textwidth}{} & %
\yz{.35\textwidth}{} \\
\midrule
\xy{.15\textwidth}{} & %
\yz{.2\textwidth}{} & %
\yz{.3\textwidth}{} & %
\yz{.35\textwidth}{} \\
\bottomrule
\end{longtable}
}

\section{Sammanfattning}

\begin{itemize}

\item Det finns flera olika tekniker som du kan använda för att hävda dig själv konstruktivt. De är bland annat grundläggande, empatisk, konsekvensbaserad och diskrepansbaserad självhävdelse, samt tekniken repig grammofonskiva och självhävdelse vid negativa känslor.

\item Det är viktigt att du tänker på din icke-verbala kommunikation när du använder de här teknikerna.

\end{itemize}

\nextup{I nästa kapitel introducerar vi olika tekniker som är bra för att sänka den allmänna spännings\-nivån i kroppen. Det hjälper dig att kommunicera konstruktivt.}

\chapter{Sänka spänningsnivån}\label{chap5}

\section{Avspänning ger bättre kommunikation}

När vi kommunicerar på ett passivt eller aggressivt sätt har vi ofta ångest eller är arga, och det återspeglas i våra kroppar: Vi kan bli spända i axlarna, nacken, käken eller i hela kroppen. Med tiden kan spänningen i kroppen öka till en punkt då vi får huvudvärk, ryggskott eller magproblem -- faktiskt en hel rad olika fysiska problem. Det blir också allt svårare att slappna av. För att kommunicera på ett sätt där vi hävdar oss själva konstruktivt behöver vi dämpa graden av aktivering i kroppen, liksom spänningsgraden som vi kan erfara när vi hamnar i svåra situationer eller lägen där vi känner oss obekväma.

\section{Identifiera fysisk spänning}

Det första steget för att reducera fysisk spänning är att identifiera var du är spänd i kroppen. Titta i listan med kryssrutor nedan. Kryssa för varje kroppsdel som är spänd just nu.

\newcommand\tickbox{%
\raisebox{-.25\baselineskip}{%
\begin{tikzpicture}
\draw (0,0) rectangle (\baselineskip,\baselineskip);
\end{tikzpicture}%
}%
}


\setlength{\columnsep}{0pt}
  \begin{multicols}{2}
    \begin{itemize}\setlength\leftmargin{0pt}\setlength\labelindent{0pt}\setlength\labelwidth{10mm}\setlength\labelsep{.75\baselineskip}\setlength\itemindent{0pt}
\item[\tickbox{}] Huvudsvålen (skalpen)
\item[\tickbox{}] Pannan
\item[\tickbox{}] Ögonen
\item[\tickbox{}] Tinningarna
\item[\tickbox{}] Käken
\item[\tickbox{}] Nacken
\item[\tickbox{}] Axlarna
\item[\tickbox{}] Bröstet
\item[\tickbox{}] Överarmarna
\item[\tickbox{}] Underarmarna
\item[\tickbox{}] Händerna
\item[\tickbox{}] Magen
\item[\tickbox{}] Ländryggen (svanken)
\item[\tickbox{}] Rumpan
\item[\tickbox{}] Låren
\item[\tickbox{}] Vaderna
\item[\tickbox{}] Fötterna
    \end{itemize}
  \end{multicols}


% {\renewcommand{\arraystretch}{1.8}
% \begin{longtable}[]{@{}RC@{}}
% \toprule
% {\sffamily\bfseries\normalsize\strut\centering Kroppsdel\strut} & %
% {\sffamily\bfseries\normalsize\strut\centering Är den spänd?\strut} \\
% \midrule
% {Huvudsvålen} & %
%  \\
% \midrule
% {Pannan} & %
%  \\
% \midrule
% {Ögonen} & %
%  \\
% \midrule
% {Tinningarna} & %
%  \\
% \midrule
% {Käken} & %
%  \\
% \midrule
% {Nacken} & %
%  \\
% \midrule
% {Axlarna} & %
%  \\
% \midrule
% {Bröstet} & %
%  \\
% \midrule
% {Överarmarna} & %
%  \\
% \midrule
% {Underarmarna} & %
%  \\
% \midrule
% {Händerna} & %
%  \\
% \midrule
% {Magen} & %
%  \\
% \midrule
% {Ländryggen} & %
%  \\
% \midrule
% {Rumpan} & %
%  \\
% \midrule
% {Låren} & %
%  \\
% \midrule
% {Vaderna} & %
%  \\
% \midrule
% {Fötterna} & %
%  \\
% \bottomrule
% \end{longtable}
% }

Se om du kan upptäcka några mönster av spända områden. Är du mest spänd i magen och ländryggen (nedre delen av ryggen), i dina armar och ben, eller runt nacken och axlarna? Du kan upprepa övningen när du är i en mer stressande situation för att se om du då är spänd i samma eller andra kroppsdelar.

\section{Lär dig sänka din fysiska spänningsnivå}

När människor börjar uppmärksamma sina kroppar blir de ofta förvånade över hur spända de verkligen är. Den goda nyheten är att det finns flera avslappningstekniker som du kan använda för att sänka den fysiska spänningsnivån i kroppen. Precis som med alla färdigheter som du lär dig i de här kapitlen gäller att ju mer du tränar på dem, desto större inverkan kommer de ha på ditt välmående.

Här är några av alla tekniker för att sänka spänningsnivån i kroppen:

\begin{itemize}

\item Träning

\item Massage

\item Progressiv muskelavslappning

\item Yoga

\item Meditation

\item Medveten närvaro-övningar

\item Andningsövningar

\item Tai chi

\end{itemize}

Kan du komma på några aktiviteter som hjälper dig att sänka spänningsnivån i kroppen? Skriv upp dem här nedan. Det kan till exempel vara tekniker som vi tagit upp här ovan eller andra personliga avslappningsfavoriter, som att lyssna på musik eller att ta ett varmt bad. Medan du arbetar med återstående kapitel, se om du kan göra någon av dessa spänningsreducerande aktiviteter och övningar varje dag.

\xlines{4}

I återstoden av kapitlet kommer vi introducera några tekniker som du kanske inte har hört talas om förut.

\section{Progressiv muskelavslappning}

Med progressiv avslappning arbetar du dig genom varje muskelgrupp i ordning. Om vi har gått omkring och varit spända en längre tid och sedan försöker slappna av, kan det vara svårt att lyckas. Pröva själv, så får du se: Försök att genast slappna av i axlarna! De flesta tycker att det är väldigt svårt att bli riktigt avslappnade på kommando.

Progressiv avslappning fungerar lite annorlunda: Först spänner du muskelgrupperna, sedan slappnar du av i dem. Genom att först spänna muskeln lär du dig att ta kontroll över den och känna igen hur det känns att vara spänd. 

Försök att spänna dina axlar: Lyft dina axlar så högt du kan, som om du ville nå dina öronspetsar med dem. Håll axlarna så i 10 sekunder. Låt sedan axlarna sjunka ned och upplev känslan av avslappning. Med progressiv avslappning gör du på samma sätt med var och en av de stora muskelgrupperna -- först spänner du dem, sedan slappnar du av.

Följ varje steg nedan för var och en av muskelgrupperna. När du går genom dem, upprepa samma mönster av spänning och avslappning. Spänn muskelgruppen i 10 sekunder. Släpp sedan spänningen och vänta i 20 sekunder innan du spänner samma muskelgrupp igen. Lägg märke till hur avslappningen känns! Det är viktigt att du slappnar av i en muskelgrupp minst dubbelt så länge som du spänner den. Poängen med den här övningen är inte spänningen utan avslappningen. Lägg noga märke till hur den känns jämfört med spänningen.

Du kanske märker att du andas in när du spänner musklerna och andas ut när du slappnar av. Det är den naturliga rytmen i kroppen och det är enklast att bara följa med i den andningsrytmen. Andas in när du spänner musklerna, andas ut när du slappnar av. 

Om du vill kan du göra en inspelning av dig själv med mobiltelefonen när du ger instruktionerna för de olika muskelgrupperna, så behöver du inte hålla ett öga på texten när du gör övningen.

\subsection{Spänning och avslappning, muskelgrupp för muskelgrupp}

Till att börja med, sätt dig bekvämt i en stol. Det är bättre att göra övningarna sittandes än liggandes, så att du inte råkar somna. Hela övningen bör ta cirka 15 till 20 minuter. Under övningen, se till att du slappnar av dubbelt så länge som du spänner dig.

Sätt båda fötterna på golvet och låt händerna vila i knät. Låt dina ögonlock falla ned så att dina ögon sluts lätt. Lägg märke till tyngden av din kropp, ditt huvud, dina axlar, dina armar och dina ben.

\begin{enumerate}

\item 
Knyt höger hand, och lägg märke till spänningen medan du gör det. Kläm åt lite hårdare och känn spänningen i näven, handleden och underarmen. Släpp taget och låt din hand slappna av. Notera skillnaden i spänning och hur avslappningen känns. Upprepa detta med höger hand en gång. Upprepa sedan allt igen, med vänster hand. Spänn och slappna sedan av i båda händerna samtidigt.

\item 
Gå vidare till underarmen, genom att böja din hand framåt och nedåt, som om du försökte nå undersidan på din arm. Låt sedan musklerna slappna av genom att räta upp handleden igen. Lägg märke till hur avslappningen känns. Upprepa med den andra underarmen.

\item 
Spänn dina båda biceps-muskler genom att pressa underarmarna upp mot dem. Spänn dem så hårt du kan. Känn spänningen. Slappna sedan av genom att räta ut armarna. Lägg märke till hur avslappningen känns. Repetera.

\item 
Flytta uppmärksamheten till axlarna. Spänn dina axlar genom att lyfta upp dem, som om du försökte nudda öronen med dem. Slappna sedan av genom att låta dem sjunka ned igen, uppmärksamma avslappningen. Upprepa en gång. 

\item 
Med armarna hängande utmed sidorna, tryck axlarna framåt. Håll så i 10 sekunder och slappna sedan av genom att låta axlarna gå tillbaka till sitt naturliga läge. Lägg märke till hur avslappningen känns.

\item 
Låt uppmärksamheten gå till din nacke. Om du har ont i nacken eller har dragit på dig en nackskada, fråga först din läkare eller fysioterapeut innan du gör den här övningen. Luta försiktigt huvudet till vänster tills dess att du känner att musklerna på höger sida av nacken spänns. Rulla sakta huvudet framåt och till höger, och sedan sakta igen tillbaka till vänster. Den här övningen ska inte göra ont. Om du känner smärta sträcker du musklerna för mycket, då behöver du ta det lite försiktigare. Upprepa en gång.

\item 
Flytta uppmärksamheten till huvudet. Gör pannan så rynkig som det går. Det kan hjälpa att lyfta på ögonbrynen. Håll så i 10 sekunder, och slappna sedan av. Lägg märke till avslappningen.

\item 
Slut dina ögon hårt och håll spänningen i 10 sekunder. Slappna av och känn hur det känns. Upprepa en gång. 

\item 
Bit samman käkarna i 10 sekunder. Släpp det hårda bettet och låt käkmusklerna slappna av. När du slappnar av helt i käken är dina läppar något särade. Lägg noga märke till skillnaden mellan muskelspänning och -avslappning. Upprepa en gång.

\item 
Pressa tungan hårt mot gommen. Slappna sedan av och låt tungan falla ned igen. Upprepa.

\item 
Pressa samman läpparna och forma dem till ett ''O''. Slappna av i läpparna. Repetera. Se till att resten av ansiktet fortfarande är avslappnat.

\item 
Flytta uppmärksamheten till bröstet. Andas in och fyll dina lungor helt samtidigt som du spänner bröstmusklerna. Håll andan i 10 sekunder innan du andas ut helt och låter musklerna slappna av. Töm lungorna helt innan du tar nästa andetag. Lägg märke till hur avslappningen känns. Repetera en gång.

\item 
Gå vidare till magen. Spänn magmusklerna och håll. Lägg märke till hur spänningen känns och slappna sedan av. Placera din hand på magen. Andas in djupt, med magen, så att handen på magen rör sig framåt. Håll och slappna sedan av. Upprepa en gång. 

\item 
Forma ryggen till en båge samtidigt som du lutar huvudet framåt. Känn spänningen i ländryggen, den nedre delen av ryggen. Håll och slappna sedan av, och räta på dig till sittande ställningen igen. Upprepa en gång. 

\item 
Spänn skinkorna i rumpan genom att pressa ihop dem. När du gör det ska du komma lite högre upp i stolen. Håll och slappna sedan av. Lägg märke till hur avslappningen känns. Upprepa en gång.

\item 
Fortsätt genom att flytta uppmärksamheten till benen. Spänn låren genom att trycka ned hälarna i golvet, så hårt du kan. Håll och slappna sedan av. Upprepa en gång. 

\item 
Spänn dina vader genom att lyfta tårna upp mot smalbenet. Håll och slappna sedan av. Upprepa en gång. 

\item 
Spänn dina fötter genom att rulla in tårna mot golvet. Håll och slappna sedan av. Lägg märke till hur avslappningen känns. Upprepa en gång.

\end{enumerate}

När du har gått genom alla muskelgrupper är det en god idé att sitta still några minuter för att verkligen känna hur det känns att vara avslappnad. Res inte på dig för snabbt, eftersom det kan göra att du återgår till det spända läget. När du slappnar av sjunker dessutom ditt blodtryck, så om du reser dig för snabbt kan du bli lite yr.

\section{Snabbavslappning}

% \subsection{Tillämpad avslappning}
% 
% [Lägga till tillämpad avslappning här? Eller ersätta snabbavslappning 1 med TA enligt \cite{ost2006tillampad}]

\subsection{Snabbavslappning 1}

Ju mer du tränar på progressiv avslappning, desto bättre kommer du bli på att hitta de delar av kroppen som är spända även i vardagslivet. Ibland hamnar du i situationer som inte har utrymme för 15 till 20 minuters avslappningsövning. För de tillfällena kan du modifiera övningen genom att välja endast de muskelgrupper som du känner är spända, och göra de motsvarande avslappningsövningarna.

\subsection{Snabbavslappning 2}

Du kan också testa den här snabba avslappningsövningen:

\begin{enumerate}

\item 
Ställ dig upp.

\item
Andas långsamt med magen, med käken avslappnad och läpparna lätt isär.

\item 
Andas in i 2 sekunder, andas ut i 3 sekunder. Lägg märke till din andning under hela övningen. Håll andningen långsam och djup.

\item
\label{xstart}Sätt ett av dina ben framför det andra, med båda fötterna stadigt på underlaget. Försök att sätta fötterna så att tårna är lika långt fram.

\item
Sätt samman händerna bakom ryggen, och vrid dem så att handflatorna i de knäppta händerna är riktade mot golvet.

\item
Håll dina händer samman på det sättet, med raka armar, och lyft försiktigt armarna bakom ryggen upp mot huvudet.

\item
Lägg märke till hur spänningen ökar i \textit{alla} dina muskelgrupper. Håll i 5 sekunder.

\item
\label{xfinish}Sätt dina fötter som vanligt, bredvid varandra, och låt armarna hänga utmed sidorna. Ta två eller tre andetag och låt spänningen släppa i hela kroppen.

\item
Upprepa steg \ref{xstart}--\ref{xfinish} tills du känner dig helt avslappnad.

\end{enumerate}

\section{Lugn, kontrollerad andning}

Sätt dig bekvämt utan korsade armar eller ben, med båda fötterna på golvet och händerna mjukt vilande i knät. Om du föredrar att sluta ögonen under den här övningen kan du göra det.

\begin{enumerate}

\item Använd näsan istället för munnen när du andas. Börja med att bara observera din andning. Lägg märke till hur det känns att andas in. Och hur det känns att andas ut.

\item Försök att sakta ned på andningstakten genom att mjukt och stadigt dra in luft ned mot magtrakten. Pausa helt kort innan du lägger märke till hur den långsamma och stadiga utandningen känns.

\item När du är uppmärksam på din andning kan du börja hitta en rytm där du räknar till tre sekunders inandning och två sekunders utandning. Det ger en andningstakt på 10 andetag i minuten. (En normal andningstakt är 10--14 andetag i minuten.)

\item Försök behålla här långsamma, jämna och kontrollerade rytmen på din andning genom att räkna tyst för dig själv i huvudet. Skynda inte på när du räknar.

\item Fortsätt andas med den här rytmen i 10 till 15 minuter. När du är klar, öppna ögonen och låt det gå en liten stund innan du lämnar stolen.

\end{enumerate}

Om du lider av panikångest är lugn och kontrollerad andning en särskilt viktig teknik för att inte hyperventilera (överandas). Andningsövningar för panik är till och med ännu långsammare än i den här övningen. Om du har panikattacker, titta närmare på självhjälpsboken \citetitle{carlbring2013ingen} av \textcite{carlbring2013ingen}. Du kan läsa och göra övningarna i den på egen hand, eller ta hjälp av en KBT-psykolog eller -psykoterapeut.

\section{Riktad visualisering}

Med tekniken visualisering använder du din egen fantasi för att sänka stressnivån. Det finns många olika typer av tekniker för visualisering. Pröva gärna en av alla ljudinspelningar och appar med guider för visualisering. Det gäller att du utforskar tills du hittar en övning som passar och är till hjälp för just dig. Vägledd visualisering praktiseras och forskas i på cancer- och smärt-kliniker runt om i världen. Visualisering är effektiv för behandling av många stressrelaterade symtom som huvudvärk, muskelkramper, kronisk smärta och ångest. Ibland kan effekterna kännas direkt och ibland tar det flera veckor av övningar innan en förbättring kommer.

Visualisering går ut på att föreställa sig en scen eller en bild så fullständigt som du kan. Ibland använder människor visualisering för att sätta mål. Till exempel kan idrottsutövare använda visualisering som en del av träningen. De kan föreställa sig ett lopp eller en tävling som de ska delta i, in i minsta detalj: De ser sig själva i loppet och föreställer sig allt de kommer att se, smaka, höra och lukta. De fantiserar om de svåra delarna av loppet och hur de kommer att klara av dem. De känner utmattningen och föreställer sig att de klarar av att hantera den.

För avslappning kan du använda en mer riktad, eller vägledd, form av visualisering. En vanlig version av det här är att föreställa dig själv på en av dina favoritplatser, eller på en vacker plats i fantasin. Även här handlar det om att föreställa sig så mycket detaljer som möjligt -- det du ser omkring dig, färgerna, temperaturen, ljuden, lukterna och känslorna i kroppen. På den här platsen känner du dig helt trygg och avslappnad. Föreställ dig själv hur du slappnar av fullständigt på den här platsen. Här finns ingen problem att lösa, inget arbete -- bara avslappning.

Det finns inga begränsningar för vad du kan använda visualisering till. Det finns dock några grundläggande principer. De är:

\begin{enumerate}

\item Släpp lite på tajta kläder, eller klä dig bekvämt. Lägg dig ned på en lugn plats och slut dina ögon mjukt.

\item ''Skanna'' din kropp: Rikta uppmärksamheten genom kroppsdel för kroppsdel och lägg märke till hur varje del känns. Gå från topp till tå eller tå till topp. Lägg märke till vilka muskler som är spända. Försök slappna av i de musklerna, så mycket det går.

\item Börja med att skapa en mental bild som involverar alla dina sinnen: synen, hörseln, känseln samt lukt- och smaksinnet. Tänk dig till exempel att du är på en strand. Se färgerna i vattnet, himlen och sanden. Se människorna omkring dig, lägg märke till vad de har på sig och hur de ser ut. Lukta på havet och den rena, friska luften. Känn värmen av solen och den milda brisen mot din hud, känn sanden mellan tårna. Hör vågorna, ljudet av fåglarna och människorna runt omkring. Smaka på saltet i luften.

\item Använd jag-budskap till dig själv för att rikta avslappningen. Använd presens och undvik negationer. Undvik till exempel att säga ''Jag är inte spänd'', vilket innehåller en negation ('inte'). Testa istället till exempel ''Jag slappnar av fullständigt''.

\item Gör visualiseringsövningen tre gånger om dagen. Allra enklast är det att göra övningen på morgonen och på kvällen.

\end{enumerate}

När du har övat på visualisering ett slag kommer du kunna använda tekniken för att sänka spänningsnivån i de flesta situationer.

\section{Sammanfattning}

\begin{itemize}

\item För att du ska kunna hävda dig själv behöver du bli medveten om när du blir fysiskt spänd och fysiskt uppjagad och lära dig att nedreglera kroppsreaktionerna. För mycket fysisk spänning kan öka den upplevda stressnivån. I extremfall kan det leda till smärta och stressrelaterade sjukdomar.

\item Det finns flera sätt att sänka spänningsnivån, inklusive meditation, träning, tai~chi, massage, visualisering, yoga, progressiv avslappning och långsam medveten andning.

\item För att minska stressnivån är det viktigt att vi övar på avslappningstekniker på daglig basis. Utforska och pröva dig fram bland de många olika teknikerna för att hitta en som passar just dig.

\item Du kan också hitta avslappningsövningar i strömmade musiktjänster eller i appar till mobilen. Ett tips för att snabbt komma igång: Spela in dig själv med mobilen när du läser instruktionerna!

\end{itemize}

\nextup{I nästa kapitel tittar vi på varför det ibland är svårt att säga 'nej', och hur vi kan bli bättre på det. Det är viktigt för att minska stressen i livet.}

\chapter{Säga 'nej' konstruktivt}\label{chap6}

\section{Att säga 'nej' i underskott}

\index{underskott}

Många personer har stora svårigheter att säga 'nej' till andra. Även personer som i andra situationer kan vara bra på att hävda sig själva kan märka att de hamnar i situationer där de säger 'ja' till saker som de egentligen inte vill ställa upp på. Nu bör det sägas att det i vissa situationer kan vara lämpligt att säga 'ja', fast du helst inte vill det. Om din chef till exempel ber dig om något väldigt viktigt som du är anställd för att göra, är det nog inte det bästa tillfället att öva självhävdelse i. Då kanske du riskerar att få sparken. Det vi menar här är andra situationer där du märker att du säger 'ja' mot din vilja. Ett exempel kan vara att en god vän ber om hjälp med något som är väldigt besvärligt för dig eller som du egentligen inte har tid för och du ändå säger 'ja'. Eller att du säger 'ja' till alla möjliga slags extra arbeten och uppgifter till den grad att du blir överbelastad.

\subsection{Vanliga följder av att inte kunna säga 'nej'}

Om du har för vana att säga 'ja' när du egentligen vill säga 'nej', kan det leda till att du bygger upp ilska och förakt mot den person du sagt 'ja' till -- även om den personen inte har gjort något fel. Det kan också göra att du känner dig alltmer frustrerad och besviken på dig själv. Om du tar på dig mer än du orkar med kan du bli överarbetad och stressad. På lång sikt kan bristande självhävdelse av det här slaget leda till sämre självkänsla, depression och ångest.

Det omvända kan också gälla: Kanske är du bara alltför bra på att säga 'NEJ!' -- men på ett sätt som är aggressivt och utan att du tar hänsyn till den andra personens rättigheter. Det kan resultera i att den andra personen blir arg eller sårad  och kanske tar avstånd från dig.

Inget av alternativen ovan är exempel på konstruktiv självhävdelse.

\subsection{Hindrande antaganden, som gör det svårt att säga 'nej'}

Vi konstaterade i kapitel~\ref{chap3}, \textit{\nameref{chap3}} att vi föds med en självklar förmåga att hävda oss själva. Alla som någon gång har umgåtts med en treåring vet att treåringar inte har några som helst problem att säga 'nej'. Under vår uppväxt lär vi oss dock nya sätt att svara på vår omgivning och våra erfarenheter kanske inte längre säger att det är lämpligt att säga 'nej'. Det kan resultera i att vi skaffar oss olika slagsövertygelser om att det är dåligt för oss att göra det. Några av dessa hindrande tankar och övertygelser finns i listan nedan. Läs den och se om några av dem stämmer för dig.

%\subsubsection{Hindrande tänkande om att säga 'nej'}

\begin{itemize}

\item ''Att säga 'nej' är oförskämt och aggressivt.''

\item ''Att säga 'nej' är inte snällt, det visar att jag är självisk och inte bryr mig om andra.''

\item ''Om jag säger 'nej' kommer den andra personen att bli upprörd och känna sig avvisad.''

\item ''Om jag säger 'nej' till någon kommer den personen inte längre att tycka om mig.''

\item ''Andras behov är viktigare än mina behov.''

\item ''Jag måste alltid försöka behaga och hjälpa andra.''

\item ''Att säga 'nej' till småsaker är larvigt och småsint.''

\end{itemize}

Kan du komma på några fler?

\xlines{5}

\section{Goda skäl att kunna säga 'nej' oftare}

De hindrande tankarna ovan är inte fakta. De är helt enkelt bara tankar och åsikter som vi har lärt oss. Var och en av dem kan vägas mot en mer hjälpsam tanke eller åsikt om vikten av att kunna säga 'nej'. I listan nedan finns några av dem.

%\subsubsection{Tänkande som underlättar att säga 'nej'}

\begin{itemize}

\item ''Andra människor har rätt att fråga, och jag har rätt att neka en fråga.''

\item ''När jag säger 'nej' nekar jag till en begäran, inte personen framför mig.''

\item ''När jag säger 'ja' till en sak säger jag faktiskt också 'nej' till något annat. Jag har alltid en valmöjlighet, och jag gör val hela tiden.''

\item Människor som har svårt att säga 'nej' överskattar ofta svårigheten som den andra personen skulle ha att acceptera ett nekande svar. Vi litar inte på att de klarar av att höra ett 'nej'. Genom att vi uttrycker våra känslor öppet och fritt kan vi faktiskt hjälpa den andra personen att uttrycka \textit{sina} känslor. När vi säger 'nej' till någon visar vi att också de kan säga 'nej' till dina frågor -- samtidigt som frågan även fortsättningsvis är fri.

\end{itemize}

Se om du kan finna ytterligare hjälpsamt tänkande. Utifrån dina egna erfarenheter, vad skulle nyttan kunna vara av att klara av att säga 'nej'?

\xlines{5}

Ibland kan du behöva fylla i en tankedagbok eller göra ett beteendeexperiment för att komma på en alternativ, mer hjälpsam tanke. Du kan använda teknikerna som du lärde dig i kapitel~\ref{chap3} för att hitta balanserande tankar om nyttan med att säga 'nej'.%, precis som du kunde använda dem för att utmana tänkandet om vikten av att inte hävda dig själv. 

Kanske märker du att du har svårt att tro på dina nya tankar. Det är helt normalt. Du har förmodligen tänkt dina gamla tankar väldigt länge, så att de numera kommer helt automatiskt för dig. Det kommer ta lite tid innan det nya tänkandet blir lika automatiserat och självklart som det gamla har varit. Fortsätt öva så kommer de nya tankarna att automatiseras.

\section{Lär dig säga 'nej'}

Vi har nu gått genom något av det tänkande som kan hindra dig från att säga 'nej'. Men även om du har koll på hur du tänker, kanske du inte vet hur du ska handla. Det finns några grundläggande principer som du kan tillämpa när du vill säga 'nej'. 

\subsection{Olika sätt att säga 'nej'}

Det finns flera sätt att säga 'nej' på. Vissa är mer lämpliga än andra i specifika situationer. I boken \citetitle{powell1997free} av \textcite{powell1997free} beskrivs sex olika sätt att säga 'nej' på. När du vill säga 'nej' kan du välja bland följande:

\begin{enumerate}

\item \textbf{Ett direkt 'nej'.} När någon ber dig att göra något som du inte vill, säg bara 'nej'. Målet är att säga nej, utan att be om ursäkt för det. Den andra personen äger problemet och du behöver inte låta honom eller henne föra över det på dig. Den här tekniken kan vara mycket kraftfull och effektiv med påstridiga säljare.

\item \textbf{Ett inkännande 'nej'.} Den här tekniken innebär att du erkänner motpartens känslor och innehållet i hans eller hennes begäran, samt lägger till ditt självhävdande avslag på slutet. Ett exempel: ''Jag vet att du vill prata med mig om hur vi ska lägga upp den årliga revisionen, men jag kan inte komma loss till lunchen idag.'' Ett annat exempel: ''Jag vet att du ser fram emot att promenera i eftermiddag, men jag kan inte komma.''

\item \textbf{Ett förklarande 'nej'.} Med den här varianten ger du helt kort ditt genuina skäl till att du säger 'nej', till exempel ''Jag kan inte äta lunch med dig idag eftersom jag måste bli klar med en rapport till imorgon.''

\item \textbf{Ett uppskjutande 'nej'.} Det här är faktiskt inte ett definitivt 'nej'. Det är ett sätt att säga 'nej' till förfrågan just nu, men som håller öppet för att säga 'ja' i framtiden. Använd tekniken endast om du genuint vill möta begäran. Exempel: ''Jag kan inte äta lunch med dig idag, men jag skulle kunna någon dag nästa vecka''.

\item \textbf{Ett undersökande 'nej'.} Precis som med den uppskjutande varianten är det är inte ett slutgiltigt 'nej'. Med det här sättet öppnar du begäran för att se om det finns något annat sätt att möta den, till exempel så här: ''Är det någon annan dag som du skulle vilja luncha?''

\item \textbf{Ett repig grammofonskiva-'nej'.} Det här sättet kan användas i många olika situationer. Du upprepar din enkla vägran, om och om igen. Du ger ingen förklaring. Du bara repeterar din vägran. Det är särskilt effektivt med envetna frågor. %Du bara repeterar din vägran. 
Till exempel: 

\begin{description}

\item[David] -- Nej, jag kan inte äta lunch med dig.

\item[Karin] -- Men kom igen, det går snabbt.

\item[David] -- Nej, jag kan inte äta lunch med dig.

\item[Karin] -- Åh, kom igen, jag bjuder.

\item[David] -- Nej, jag kan inte äta lunch med dig.

\end{description}

\end{enumerate}

\section{Sammanfattning}

\begin{itemize}

\item Att säga 'nej' kan vara svårt för många människor.

\item Som små barn har vi inga problem att säga 'nej', men när vi lär oss av omgivningen och våra erfarenheter kan vi utveckla svårigheter att göra det.

\item Att säga 'ja' när vi egentligen menar 'nej' är ett recept på stress, självförakt och ilska.

\item Om vi har svårt att säga 'nej' är det ofta för att vi har ett antal hindrande tankar och antaganden om att göra det. Ett sätt att frigöra sig från dem är att upptäcka att de är åsikter snarare än fakta. Du kan använda tankedagböcker och beteendeexperiment (se kapitel~\ref{chap3}) för att utmana och balansera dina hindrande tankar och antaganden.

\item Några goda råd för att göra det lättare att säga 'nej': håll ditt budskap kort, var tydlig och var ärlig.

\item Det finns många sätt att säga 'nej' på. De inkluderar ett direkt nej, ett undersökande nej, ett uppskjutande nej, ett förklarande nej och ett repig grammofonskiva-nej.

\end{itemize}

\nextup{I nästa kapitel går vi genom hur vi kan hantera kritik på ett konstruktivt sätt. Att kunna ta och ge kritik på ett konstruktivt sätt är bra för självkänslan.}

\chapter{Hantera kritik konstruktivt}\label{chap7}

\section{Vad är 'kritik'?}

Vi har alla blivit kritiserade i livet. Att klara av att ta emot kritik på ett konstruktivt och självhävdande sätt är en av de viktigaste saker vi ställs inför på vår utvecklingsresa mot allt större grad av mognad. Ordet 'kritik' kommer från ett gammalt grekiskt ord som beskriver en person som sakligt ger ett omdöme, en analys, en värdering, en tolkning eller en observation till någon annan. För att kunna acceptera kritik på ett moget sätt behöver vi alltså kunna ta emot återkoppling på vårt handlande från andra. Denna återkoppling kan komma i form av analyser, observationer eller tolkningar.

\subsection{Olika slags kritik}

Kritik kan antingen vara konstruktiv eller destruktiv. Konstruktiv kritik är utformad för att ge genuin återkoppling på ett hjälpsamt och icke-hotande sätt, för att personen som får kritiken ska kunna lära sig något nytt och växa i något visst avseende. Återkopplingen är typiskt valid, det vill säga att det är en sann kritik, exempelvis ''Jag tyckte mycket om hur du skrev rapporten; jag tror att den kan bli ännu bättre om du ägnar tid åt att förbättra din stavning.''

Destruktiv kritik är varken valid eller sann, eller så är den valid men ges på ett sätt som är extremt ohjälpsamt. Den destruktiva kritiken ges ofta av någon som inte har tänkt efter ordentligt, eller som vill skämma ut eller skada mottagaren, exempelvis ''Den här rapporten är gräslig, din stavning är förskräcklig.''

\section{Varför reagerar vi som vi gör?}

Hur vi tar emot kritik hänger till stor del samman med våra upplevelser av att bli kritiserade som barn. Om vi inte har några tidigare erfarenheter av att bli kritiserade kan vi bli helt förkrossade när vi för första gången får kritik. Om vi som barn fick mycket konstruktivt kritik klarar vi kanske av att ta emot kritik väl när vi växer upp. Om vi fick hård och bestraffande kritik, kan vi däremot uppfatta all slags kritik som sårande och avvisande.

Det senare händer ofta när kritiken handlade om oss som personer snarare än om våra beteenden.
Om du till exempel som barn gjorde ett misstag och fick höra någon säga ''du är dålig'', säger kritiken underförstått att det var du som person som var dålig. Sådan kritik är avvisande och kan få dig att känna att förändring är utsiktslös. Om du däremot fick veta att ''det där var en dålig sak du gjorde'', så är det lättare att skilja på beteendet och personen. Det var beteendet som var dåligt, inte du som person. Därmed står det också i din makt att förändra ditt handlande.

\subsection{Hur tar du kritik?}

Hur tar du emot kritik från andra? Här är några vanliga sätt att svara på kritik, som inte är exempel på självhävdelse:

\begin{itemize}

\item Bli förvirrad

\item Bli defensiv

\item Larva sig

\item Ignorera kritiken utåt men bli sårad inombords

\item Internalisera ilska och koka inombords

\item Hämnas med ilska och skuldbeläggande

\item Stänga ned

\item Dra sig undan

\item Springa därifrån

\end{itemize}

Tänk tillbaka på när du senast själv blev kritiserad. Skriv ned vad det var för situation och hur du reagerade.

Situationen:\par\nobreak\xlines{3}

Så reagerade jag:\par\nobreak\xlines{5}

Läs följande beskrivningar och se om du kan identifiera vilken responsstil som bäst stämmer in på hur du brukar reagera.

\subsection{Passiv respons på kritik}

Om vi oftast svarar med passivitet på kritik, får vi svårigheter att ta emot den konstruktivt. Kanske går vi bara iväg och känner oss sårade och förvirrade. Eller så tenderar vi att hålla med om vilken kritik som helst, oavsett om den är valid eller inte. Och så klandrar vi oss själva för det. Exempel: ''Ja, du har rätt, jag är \ldots{} jag är värdelös, jag måste lägga av.'' Då kan vi tolka kritik av vårt beteende som ett avvisande av oss själva som personer. Den typen av respons kan i sin tur leda till nedstämdhet, ångest och låg självkänsla. Det känns som att världen är en plats fylld med kritik -- och vi håller med om den!

En alternativ respons när vi får kritik är att skratta och sedan öppet kritisera oss själva ännu mer, med attityden ''Om jag kritiserar mig själv ännu hårdare och gör det på ett skämtsamt sätt kommer ingen veta hur sårad jag egentligen är.'' På lång sikt har den här responsen samma effekt som att öppet hålla med om all kritik som vi får.

\subsection{Aggressiv respons på kritik}

Om vi brukar svara aggressivt betyder det att vi tolkar kritik som en attack på våra personer. När vi uppfattar att vi blir attackerade blir vi defensiva och går kanske också själva till attack. Till exempel: ''Hur vågar du?! Jag är inte sen. Det är du som alltid är sen.'' Den här typen av respons kan ofta leda till konflikt och aggression, vilket i sin tur kan leda till nedstämdhet och låg självkänsla.

\subsection{Konstruktiv självhävdelse vid kritik}

Att hävda sig själv vid kritik handlar om att identifiera och se skillnad på konstruktiv och destruktiv kritik, och att handla därefter. När vi hävdar oss själva konstruktivt förstår vi att kritiken inte nödvändigtvis handlar om oss själva som personer. Vi blir inte defensiva, arga, sårade eller klandrar oss själva. Vi springer inte därifrån. Vi behåller lugnet och tar emot kritiken utan negativa känslouttryck. I nästa avsnitt finns sätt att tänka som gör det enklare att svara på kritik med konstruktiv självhävdelse.

\section{Tänkandet påverkar hur du agerar}

\subsection{Hindrande sätt att tänka}

Som med andra icke självhävdande beteenden finns det ofta hindrande tänkande i botten. Här är några exempel.

\begin{itemize}

\item ''Om jag blir kritiserad betyder det att jag är dum.''

\item ''Eftersom de kritiserade mig, tycker de inte om mig längre.''

\item ''De har rätt, jag gjorde fel, jag kan inte gör något rätt. Jag är misslyckad.''

\item ''Jag kan inte kritisera dem för då kommer de inte att tycka om mig.''

\item ''Hur vågar de säga att jag har gjort fel? De har ingen rätt att göra det.''

\item ''De är hur som helst idioter. Jag tänker inte lyssna på dem.''

\item ''Om jag kritiserar mig själv ännu hårdare och skojar om det kommer ingen att få veta hur sårad jag är.''

\end{itemize}

Kan du identifiera ytterligare hindrande tankar, utifrån dina egna erfarenheter, som gör det svårt för dig att möta kritik med konstruktiv självhävdelse? Skriv upp dem.

\xlines{5}

\subsection{Konstruktiva sätt att tänka}

Här är några hjälpsamma och självhävdande tankar, som du kan använda för att balansera negativa, hindrande tankar som du kan ha. Kom ihåg att du också kan använda tankedagböcker och beteendeexperiment (se kapitel~\ref{chap3}) för att komma på fler hjälpsamma, balanserande tankar.

\begin{itemize}

\item ''Om det är något fel på det jag gjort betyder det inte något om mig som person. Jag behöver skilja mellan mitt handlande och mig själv som person.''

\item ''Vad kan jag lära mig av kritiken? Det mesta av kritiken innehåller sannolikt ett korn av sanning. Kritik kan te sig negativt. Men varje gång jag får kritik har jag möjlighet att lära mig av den. Jag kan vill alltid fråga mig själv vad jag kan lära mig av kritiken.''

\item ''Jag har rätt att berätta att jag känner mig sårad, irriterad eller upprörd, av andras beteende.''

\item ''Att ge direkt återkoppling och kritik kan vara kärleksfullt och omtänksamt.''

\end{itemize}

Kan du komma på några andra konstruktivt självhävdande tankar om kritik? Om du skrev upp några egna hindrande tankar ovan, testa att identifiera balanserande tankar till dem.

\xlines{5}

\section{Ta emot konstruktiv kritik}

Vi behöver alla klara av att ta emot konstruktiv kritik. Du kan reagera på flera olika sätt, beroende på hur kritiken framförs.

\begin{enumerate}

\item \textbf{Acceptera kritiken.} Om kritiken är valid, ta emot den utan att uttrycka skuld eller negativa känslor. Acceptera att du inte är perfekt och att det enda sättet du kan lära dig något är att testa, ibland göra misstag, se vad som behöver ändras och testa igen. Tacka personen för återkopplingen, om det är lämpligt. Se kritiken som en gåva.

\item \textbf{Använd omvänd konstruktiv självhävdelse.} Den här tekniken handlar inte bara om att acceptera kritiken, utan också öppet hålla med om den. Den används när någon framför berättigad kritik till dig. Den här färdigheten handlar om att lugnt hålla med om kritiken av dina negativa egenskaper, utan att be om ursäkt eller för den skull känna dig förkrossad. En konversation kan till exempel se ut så här:

\begin{description}

\item[Kritik] -- Ditt skrivbord är väldigt stökigt. Du är desorganiserad.

\item[Svar] -- Ja, det är sant. Jag är inte så ordentlig.

\end{description}

Nyckeln till omvänd konstruktiv självhävdelse är självförtroende och övertygelsen att du har förmågan att ändra dig själv om du önskar göra det. Genom att hålla med om och acceptera kritiken, om den är befogad, så behöver du inte känna dig helt förstörd. Den här typen av svar kan också användas för att desarmera spända situationer. Om någon kritiserar dig aggressivt kan det bero på att hen förväntar sig att du ska svara aggressivt. Spänningen släpper när du håller med personen.

Ett annat sätt att använda omvänd självhävdelse på ett konstruktivt sätt är att ta ägarskap över ett misstag innan någon annan hinner påpeka det. Om du till exempel kommer för sent kan du säga ''Hej, jag är sen.''

\item \textbf{Använd omvänt, konstruktivt undersökande.} Omvänt undersökande innebär att du på ett konstruktivt sätt ber om fördjupad och mer specifik kritik. Om du blir osäker på om kritiken som framförs är berättigad eller konstruktiv, kan du be om mer detaljer. Till exempel:

\begin{description}

\item[Kritik] -- Det kommer inte bli så lätt för dig, eller hur, eftersom du är blyg?

\item[Svar] -- På vilka sätt tycker du att jag är blyg?

\end{description}

Om kritiken är konstruktiv, kan informationen som kommer när du frågar om mer detaljer användas på ett konstruktivt sätt. Då förbättras också kommunikationen mellan dig och samtalspartnen. Om kritiken däremot är manipulerande eller avsedd att såra visar du med dina frågor att den inte är konstruktiv.

\end{enumerate}

\section{Ta emot destruktiv kritik}

Tyvärr kommer vi alla att stöta på destruktiv kritik någon gång under våra liv. Destruktiv kritik är svårare att hantera än konstruktiv kritik. Om vi övar på teknikerna nedan, kan vi bli skickligare på att hantera dessa svåra situationer. Som med alla andra färdigheter i livet krävs det övning och tar en hel del tid innan du känner dig säker på dem. Lägg märke till att vissa av de här färdigheterna är samma färdigheter som används för att hantera konstruktiv kritik.

\begin{enumerate}

\item \textbf{Inte hålla med.} Den första tekniken handlar helt enkelt om att inte samtycka till kritiken. Det är viktigt att du behåller lugnet och tänker på dina icke-verbala beteenden, inklusive röstläget. Det är lätt att glida över i aggressivitet eller passivitet när du inte håller med om något. Se personen i ögonen och svara med lugn röst. Till exempel:

\begin{description}

\item[Kritik] -- Du är alltid sen.

\item[Svar] -- Nej, jag är inte alltid sen. Det händer ibland att jag är sen, men jag är verkligen inte \textit{alltid} sen.

\end{description}

\item \textbf{Använda omvänt, konstruktivt undersökande.} Som vi nämnde ovan kan det hända att du inte kan avgöra om kritiken som framförs av någon annan är konstruktiv eller destruktiv. Vi behöver förstå vad personen menar. Om kritiken är destruktiv kan vi antingen inte hålla med om den, som i exemplet ovanför, eller använda en av de avväpnande teknikerna nedan.

\item \textbf{Använda avväpnande (eller uppluckrande) tekniker.} De här teknikerna handlar om att på ett eller annat sätt göra kritiken ''luddig i kanterna'', eller att ''luckra upp den''. Syftet med dem är att desarmera en potentiellt aggressiv eller svår situation. Använd de här teknikerna när kritiken varken är välgrundad eller konstruktiv. När vi får destruktiv kritik är den vanligaste reaktionen för de flesta att antingen svara med samma mynt, det vill säga att bli aggressiva och slå tillbaka, eller att bli passiva och krypa ihop. Inget av dessa alternativ är särskilt bra.

Tricket med de här teknikerna är att hitta ett sätt att hålla med om en liten del av vad motparten säger. Genom att hålla lugnet och att inte gå med på att bli provocerad eller upprörd, kan du använda teknikerna för att dränera kritiken på dess destruktiva kraft.

Det finns tre olika slags avväpnande tekniker:

\begin{enumerate}

\item \textbf{Delvis hålla med.} Den här tekniken går ut på att du hittar en del av kritiken som är befogad, och håller med om endast den delen.

Exempel 1:

\begin{description}

\item[Kritik] -- Det går inte att lita på dig. Du glömmer hämta barnen, du betalar aldrig hyran så nu riskerar vi att bli utan tak över huvudet och du finns aldrig där för mig när jag behöver dig.

\item[Svar] -- Du har helt rätt i att jag glömde hämta barnen från simträningen förra veckan.

\end{description}

Exempel 2:

\begin{description}

\item[Kritik] -- Du jobbar inte, du gör precis ingenting.

\item[Svar] -- Ja, det är sant, jag har inget jobb.

\end{description}

\item \textbf{Hålla med, men inte om sannolikheten.} Med den här tekniken kan du hålla med om att ett utfall är möjligt, även om du själv tror att sannolikheten för det utfallet är en på miljonen.

Exempel:

\begin{description}

\item[Kritik] -- Om du inte använder tandtråd kommer du att tappa tänderna och det kommer du ångra resten av livet!

\item[Svar] -- Du har rätt, jag kan få tandlossning.

\end{description}

\item \textbf{Hålla med i princip.} Med den här tekniken håller du med om den andra personens logik, utan att hålla med om innehållet i det han eller hon säger.

Exempel:

\begin{description}

\item[Kritik] -- Det är fel verktyg för det du håller på med. Ett stämjärn kan slinta och förstöra trästycket. Du borde använda ett håljärn istället.

\item[Svar] -- Du har rätt, om stämjärnet slinter kommer det verkligen förstöra träämnet.

\end{description}

\end{enumerate}

\end{enumerate}

\subsection{Extra tips när vi blir kritiserade}

\begin{enumerate}

\item \textbf{Svara på innehållet (orden) i kritiken, inte tonen.} Det är viktigt att dela upp det som kritiken handlar om från sättet den framförs på. När människor kritiserar andra uppfattas de ofta som konfronterande eller till och med aggressiva. Det kan betyda att vi avfärdar det de säger, trots att innehållet i kritiken faktiskt kan vara användbart. Det krävs träning för att lyckas skilja på innehållet i kritiken från stilen på kritiken. Även om personer pratar på ett argt sätt, bör vi försöka att koppla loss oss från känsloläget i kritiken och förhålla oss till de användbara förslag som finns därunder.

\item \textbf{Svara inte direkt.} Det kan vara bäst att vänta lite med att svara på kritik. Om vi svarar direkt kan vi lätt göra det med arga känslor eller sårad stolthet, vilket lätt kan få konsekvenser som vi sedan ångrar. Om vi skjuter in lite tålamod ger det oss möjlighet att reflektera över kritiken i lugn och ro, innan vi svarar.

\item När du känner dig kritiserad -- gör så här:

\begin{enumerate}

\item[1.] Stanna upp. Reagera inte förrän du är säker på vad som pågår.

\item[2.] Fråga dig: har du verkligen blivit kritiserad? Eller försöker du ''tankeläsa'' den andra personen?

\item[3.] Fråga personen. Om du är osäker, fråga! Till exempel: ''Vad menade du med det där?''

\item[4.] När du har rett ut om det verkligen handlar om kritik, utvärdera om den är befogad och valid, eller inte. Svara sedan med hjälp av någon av teknikerna i det här kapitlet.

\end{enumerate}

\end{enumerate}

\section{Ge konstruktiv kritik}

Så här långt komna har vi pratat om färdigheter för att ta emot och hantera kritik. Det finns också färdigheter att träna på för att ge kritik. Dessa färdigheter handlar om att lägga fram kritiken på ett sätt som gör det möjligt för personen som får kritiken att ta emot det du vill säga. Du har all rätt att begära en ändring i någons beteende om beteendet sårar, upprör eller irriterar dig på något sätt. Kom ihåg att det inte är säkert att personen kommer att ändra sättet han eller hon beter sig på även om du ber om det. Men genom att uttrycka dina behov slipper du bygga upp harm och förbittring som skulle kunna ge ännu större problem i relationen till den andra personen. 

Att ge befogad kritik har ett värde i sig, oavsett om den efterlevs eller inte. Rak och ärlig återkoppling kan vara uttryck för både omtanke och hjälpsamhet. Det spelar ingen roll om kritiken är negativ eller positiv -- att du uttrycker den visar du den andra personen att du värderar honom eller henne, och relationen mellan er.

Försök att följa riktlinjerna nedan när du ger konstruktiv kritik.

% kl får ge mig befogad kritik för något

\begin{enumerate}

\item Välj tid och plats noga. Om du ger konstruktiv kritik om något som har orsakat en stark känslomässig reaktion, vänta tills du har lämnat den jobbiga situationen och hunnit lugna ned dig. Sedan kan du framföra din kritik. Vänta inte tills du hamnar i samma situation igen innan du konfronterar beteendet.

\item Beskriv beteendet du kritiserar, istället för att sätta ''etiketter'' på personen. Säg till exempel hellre ''Du fick med ett fel i rapporten'' än ''Är du dum i huvudet?''

\item Beskriv dina känslor med jag-budskap, utan att klandra den andra personen.  Säg till exempel ''Jag blir arg när \ldots{}'' snarare än ''Du gör mig arg.''

\item Be om en konkret förändring. Säg till exempel hellre ''Den höga musiken stör mig verkligen, vill du vara så vänlig att sänka ljudvolymen efter 20:00?'' istället för ''Jag står inte ut med din höga musik''.

\item Ange både de positiva konsekvenserna om personen efterföljer din begäran, och de negativa konsekvenserna om personen inte ändrar sitt beteende.

\item Ge realistiska förslag på förändring och realistiska konsekvenser om personen inte följer din begäran. Uttala inte tomma hot. Säg till exempel inte ''Jag slår ihjäl dig om du inte sänker volymen.''

\item Fråga personen vad han eller hon känner om det du nyss har sagt. Att hävda sig själv handlar om att ge förutsättningar för jämställd kommunikation. Var försiktig med att det här inte urartar till ett utbyte av kritik.

\item Försök avsluta i en positiv riktning. Om det är lämpligt, lägg till en positiv beskrivning om dina känslor gentemot den andra personen.

\end{enumerate}

\section{Sammanfattning}

\begin{itemize}

\item Vi blir alla kritiserade. Att kunna ta emot konstruktiv kritik är en viktig färdighet som vi kan lära oss. 

\item Kritik kan vara konstruktiv eller destruktiv.

\item Hur vi svarar på kritik kan vara ett resultat av vilket slags kritik vi har fått tidigare i livet.

\item Som med andra exempel på beteenden som inte är självhävdande, kan sättet vi tänker på göra att vi svarar med antingen passivitet eller aggressivitet när vi uppfattar att vi blir kritiserade. Vi kan rikta vårt tänkande mot konstruktiv självhävdelse.

\item Färdigheter som underlättar hantering av konstruktiv kritik är bland annat: acceptans av kritiken; omvänd, konstruktiv självhävdelse; och omvänt, konstruktivt undersökande.

\item Färdigheter som underlättar hantering av destruktiv kritik är bland annat: att inte hålla med om kritiken; omvänt, konstruktivt undersökande; och avväpnande tekniker som gör att kritiken tappar sin kraft.

\item Ett ytterligare tips är att lyssna till innehållet i kritiken, snarare än tonen den framförs på, och att vänta lite med att svara för att ge sig själv tid för eftertanke.

\item När du själv framför kritik, är det viktigt att välja tid och plats med omsorg, att kritisera beteenden snarare än personen och att vara tydlig med vilken förändring du önskar. Om möjlighet finns, avsluta gärna i en positiv riktning.

\end{itemize}

\nextup{I nästa kapitel går vi genom hur vi kan hantera besvikelser på ett sätt som respekterar både våra egna och andras rättigheter.}

\chapter{Hantera besvikelser konstruktivt}\label{chap8}

\section{Att bli besviken}

Det vore omöjligt att gå genom livet utan att bli besviken över något. Besvikelser inträffar när vi förväntar eller önskar oss att något ska hända, som sedan inte händer.

\section{Hur hanterar du besvikelser?}

Här är några sätt som människor kan hantera besvikelser på:

\begin{itemize}

\item Dra sig undan och tjura

\item Bli arg på saken eller personen som orsakat besvikelsen. Det kan inkludera att bli arg på sig själv.

\item Bli nedstämd

\item Vilja utkräva hämnd

\item Bli uppgiven

\item Kritisera sig själv

\end{itemize}

Fundera en liten stund på när du senast blev besviken. Försök att komma ihåg hur du reagerade. Skriv ned det du kommer ihåg.

Situationen du blev besviken i:

\xlines{2}

Vad gjorde du?

\xlines{5}

Titta nu genom beskrivningarna nedan för att se om det går att förstå beteendet som passivitet, konstruktiv självhävdelse eller aggressivitet.

\subsection{Passiva svar på besvikelser}

Om du reagerar på en besvikelse med passivitet är det troligt att du också ger upp dina försök att gå i den önskade riktningen i livet. Då är risken också stor att du blir överdrivet självkritisk för att du inte kunde nå dit du ville. Det kan också hända att du tycker synd om dig själv och drar dig undan för att älta. Om det är andra personer som har gjort dig besviken, kanske du ger upp om dem också. Alla dessa reaktioner kan försämra självkänslan och leda till nedstämdhet.

\subsection{Aggressiva svar på besvikelser}

Om du svarar med aggressivitet på en besvikelse känner du dig sannolikt arg på personen eller situationen som orsakade den. Risken ökar då att du blir bitter på personen eller situationen. Kanske märker du också att du blir hämndlysten.

\subsection{Konstruktiv självhävdelse vid besvikelser}

Om du svarar med konstruktiv självhävdelse på en besvikelse, är det mycket möjligt att du fortsätter att känna dig besviken över att saker inte har gått din väg. Det är normalt! Det som är skillnaden mot de andra svarsstilarna är att du varken klandrar dig själv eller någon annan för hur saker har utvecklat sig. Det innebär också att du inte fastnar i ett negativt känsloläge. Du tar ansvar för din del i besvikelsen och tänker igenom vad du kan göra för att komma vidare från det nuvarande läget. Det kan innebära att du måste ändra på något i livet -- att det kan finnas något som du kan lära dig av din besvikelse. Eller så finns det ingenting alls som du kan förändra. Oavsett så handlar konstruktiv självhävdelse om att acceptera hur det har gått.

\section{Tänkandet påverkar hur du agerar}

Precis som med andra beteenden som vi har tittat på finns det både hindrande och konstruktiva sätt att tänka, som i sin tur påverkar hur du agerar.

\subsection{Hindrande sätt att tänka}

Hur du hanterar en besvikelse hör samman med hur du tänker. Här är några exempel på hindrande tankar, som kan göra det svårt för dig att hävda dig själv konstruktivt när du blir besviken.

\begin{itemize}

\item ''De borde veta att jag inte gillar när de gör så där.''

\item ''Världen är hemsk, jag står inte ut med det här.''

\item ''Den personen är ond.''

\item ''Jag vägrar acceptera att den personen gör på det sättet.''

\item ''Jag tolererar inte det här, det är oacceptabelt.''

\end{itemize}

Kan du hitta några andra tankar som hindrar dig från att svara på besvikelser med att hävda dig själv konstruktivt?

\xlines{5}

\subsection{Konstruktiva sätt att tänka}

Här är några hjälpsamma sätt att tänka som kan balansera hindrande tankar, så att du lättare kan hävda dig själv konstruktivt när du blir besviken. Kom ihåg att du också kan använda tankedagböcker och beteendeexperiment (se kapitel~\ref{chap3}) för att utifrån dina egna tidigare erfarenheter hitta fler hjälpsamma tankar.

\begin{itemize}

\item ''Det är inte roligt att bli orättvist behandlad, men det är inte heller fruktansvärt eller katastrofalt.''

\item ''Jag kan stå ut med att vara sårad och frustrerad. Jag kan göra något för att påverka situationen.''

\item ''Jag accepterar hur den andra personen är. Han eller hon har kanske avvisat en del av mina beteenden men inte hela mig som person.''

\item ''Det är bäst att jag öppet uttrycker mina känslor; konsekvenserna av det behöver inte bli så hemska som jag tror.''

\end{itemize}

Försök hitta egna tankar om värdet av att hävda dig själv konstruktivt, även när du blir besviken. Om du hittade egna hindrande tankar ovan, försök att också hitta egna balanserande tankar till dem.

\xlines{5}

\section{Sammanfattning}

\begin{itemize}

\item Vi blir alla besvikna i livet. Det går inte att gardera sig mot besvikelser.

\item Som med annat beteenden där vi inte hävdar oss själva konstruktivt, kan vi arbeta med sättet som vi tänker. Genom att balansera de negativa tankar som far genom huvudet i situationer där vi blir besvikna, ger vi oss själva möjlighet att också svara på besvikelserna på ett mer konstruktivt sätt.

\end{itemize}

\nextup{I nästa kapitel lär vi oss hur vi både kan ta emot och ge komplimanger på ett självhävdande och konstruktivt sätt. Det är en viktig självhävdande och social färdighet.}

\chapter{Ge och ta emot komplimanger konstruktivt}\label{chap9}

\section{Komplimanger bygger relationer och självkänsla}

Att ge och ta komplimanger är en annan färdighet som har med självhävdelse att göra, eftersom det är en viktig del av vardagliga konversationer, byggande av relationer och utvecklande av självkänsla. Precis som med alla andra färdigheter som vi har fördjupat oss i, så hindrar vårt sätt att tänka ofta vår förmåga att ta emot och ge komplimanger. Det kan också vara så att vi saknar färdigheter som behövs för att i nästa steg kunna utveckla färdigheten att ta emot och ge komplimanger. Det här kapitlet hjälper dig att identifiera dina hindrande tankar, och visar också stegvis hur du kan göra för att förbättra sättet som du förhåller dig till komplimanger på.

\subsection{Ta emot komplimanger}

En del personer tycker att det är extremt svårt att ta emot komplimanger. Det är förståeligt att det ibland känns obekvämt att få beröm, men det är en viktig social färdighet för självhävdelse att konstruktivt kunna ta emot andra människors positiva kommentarer om vårt utseende, vårt arbete eller andra aspekter av oss själva.

\section{Hur hanterar du komplimanger?}

Sättet man kan ta emot komplimanger på som inte är uttryck för självhävdelse är många. De inkluderar:

\begin{itemize}

\item Ignorera komplimangen och byta samtalsämne

\item Säga emot en del av eller hela komplimangen, exempelvis ''Faktiskt tycker jag inte om färgen på klänningen alls.''

\item Avfärda eller låta komplimangen rinna av sig, till exempel ''Åh, den här gamla trasan, den är inget särskilt.''

\item Vara sarkastiskt. Till exempel: ''Javisst, den är ju \textit{helt fantastisk}, eller hur?''

\item Le eller skratta nervöst

\item Börja tala självkritiskt

\end{itemize}

När fick du senast ta emot en komplimang? Fundera en stund på det och se om du kan komma ihåg hur du reagerade. Beskriv situationen och hur du reagerade.

Situation:

\xlines{2}

Vad jag gjorde:

\xlines{5}

Var din reaktion passiv, självhävdande eller aggressiv? Läs beskrivningarna nedan för att se om du hade rätt.

\subsection{Passiva svar på komplimanger}

Om du svarar passivt på komplimanger är det sannolikt att du antingen ignorerar komplimangen helt, avleder uppmärksamheten från den eller minimerar berömmet. Möjligen gör du detta för att du känner dig nervös eller obekväm och inte vet vad du ska göra. Det kan resultera i att den andra personen också känner sig obekväm. Kanske går du då från situationen med en känsla av förlägenhet. Då kan självkänslan ta en smäll.

\subsection{Aggressiva svar på komplimanger}

Om du blir irriterad, arg eller defensiv, eller säger emot eller svarar sarkastiskt på berömmet -- då är ditt svar aggressivt. Det kan få den andra personen att känna sig obekväm, precis som vid passiva svar.

\subsection{Självhävdande svar på komplimanger}

Om ditt svar bygger på att du hävdar dig själv, och låter dig ta plats i situationen, så tar du emot komplimangen på ett positivt sätt, både för dig själv och avsändaren. Den andra personen känner sig bekväm och interaktionen leder till att ni båda mår bättre. Du mår bättre eftersom du accepterade komplimangen. Den andra personen mår bättre eftersom han eller hon fick ge dig den.

Att välja att ta emot komplimanger kräver en hel del arbete. Till att börja med måste du aktivt bortse från den självkritik som med all sannolikhet dyker upp. Den kommer automatiskt, som ett resultat av lång tids träning på att formulera självkritik. Därefter behöver du träna på att agera som om du trodde på komplimangen, även om det till en början känns krystat. När du till slut har lärt dig att omfamna positiv återkoppling och ta emot komplimanger på ett vänligt sätt öppnar det dörren för ännu fler varma tankar och positiva interaktioner. Så småningom kan du faktiskt \textit{börja tro på komplimangerna}!

\section{Tänkandet påverkar hur du agerar}

Olika hindrande tankar kan göra det svårt för oss att hävda oss själva även i situationer när vi får komplimanger. Vi kan balansera dessa hindrande tankar med konstruktiva tankar.

\subsection{Hindrande sätt att tänka}

Samma sak gäller för situationer där vi får komplimanger, som många andra situationer som vi tittar på i den här boken. Här är några av de hindrande tankarna som vi kan ha:

\begin{itemize}

\item ''De menar det inte egentligen. De försöker bara vara trevliga.''

\item ''De är inställsamma.''

\item ''De är ute efter något.''

\item ''Om jag tar emot komplimangen betyder det att jag är högfärdig.''

\item ''Om jag tar emot komplimangen tror de att jag är fåfäng.''

\item ''Det är alldeles för pinsamt för att jag ska kunna säga något tillbaka.''

\end{itemize}

Kan du identifiera några andra hindrande tankar, som gör det svårt för dig att hävda dig själv och ta plats när du tar emot komplimanger? Skriv upp dem här:

\xlines{5}

\subsection{Konstruktiva sätt att tänka}

Här är några konstruktiva tankar som du kanske samtidigt har, som du kan använda för att balansera dina hindrande tankar. Tankedagböcker och beteendeexperiment (se kapitel~\ref{chap3}) kan också komma till pass för att hjälpa dig att identifiera fler hjälpsamma tankar som ger dig möjlighet att hävda dig själv.

\begin{itemize}

\item ''Komplimangen kan vara äkta.''

\item ''Även om de bara försöker vara vänliga så är det en positiv sak och jag vill svara på ett lämpligt sätt.''

\item ''Om jag inte tar emot komplimangen kommer jag att få den andra personen att känna sig obekväm.''

\item ''Om jag tar emot komplimangen kan jag få den andra personen att också må bättre.''

\item ''Även om jag tar emot en komplimang på ett vänligt sätt behöver jag inte hålla med om den helt och hållet.''

\item ''Om jag börjar att tro på en del komplimanger kommer jag inte känna mig lika dålig längre.''

\item ''Att jag tar emot en komplimang betyder inte att jag är uppblåst eller tror att jag är bättre än andra. Om så vore fallet skulle jag ju redan ge mig själv massor av komplimanger!''

\item ''Människor ger komplimanger av massor av olika skäl. Det är inte lönt att slösa tid på att undra varför någon gjorde det mot mig. Jag vill uppskatta det faktum att någon tog sig tid att säga något snällt till mig!''

\item ''Jag har lika mycket rätt till komplimanger som någon annan.''

\end{itemize}

Kan du hitta några fler konstruktiva tankar om att ta emot komplimanger, som gör det enklare att hävda dig själv i sådana situationer? Använd dem (eller några av ovanstående, som du tror på) för att balansera de negativa, hindrande tankarna du skrev ned här ovan.

\xlines{5}

\section{Lär dig ge och ta emot komplimanger}

\subsection{Ta emot komplimanger}

\begin{enumerate}

\item Se på den andra personen. Sitt eller stå rakryggad. Om du sjunker ihop eller inte ser på personen kan det kännas som att du misstror eller inte tycker om honom eller henne.

\item Lyssna på vad personen säger.

\item Le när du tar emot komplimangen. En komplimang är avsedd att få dig att känna dig bra. Om du rynkar ögonbrynen eller tittar ned eller undan kan den andra personen bli förvirrad och känna sig obekväm.

\item Avbryt inte.

\item Säg 'tack' eller något annat som visar att du uppskattar det personen har sagt.

\item Kom ihåg att acceptera komplimangen utan att byta ämne från dig själv, och utan att tvinga dig att genast betala tillbaka. Det hjälper dig att känna dig mer säker och ger dig själv utrymme att tycka om dig själv bättre.

\end{enumerate}

\subsubsection{Fler tips för att ta emot komplimanger konstruktivt}

\begin{itemize}

\item Övning ger färdighet: Att lära sig ta emot komplimanger kräver aktiv övning. Här är en sak som du kan göra för att komma igång direkt. Ställ dig framför en spegel. Föreställ dig att någon säger något snällt till dig, säg sedan själv: ''Tack så mycket. Det betyder mycket för mig.''

\item Byt inte samtalsämne utan att först ha tagit emot och tackat för eller erkänt en komplimang.

\item Det är ofta fördelaktigt att använda sig av komplimangen för att föra konversationen vidare. ''Tack så mycket! Jag hittade den på Blocket -- det är helt otroligt vad man kan hitta där.'' På så sätt accepterar du både komplimangen och för samtalet vidare till något som ni båda kan fortsätta prata om.

\end{itemize}

\subsection{Ge komplimanger}

Det är också viktigt att lära sig att ge komplimanger. Att ge komplimanger är ett sätt att visa att du har sett en annan person och uppskattar honom eller henne eller situationen som ni är tillsammans i. Människor tycker om att vara tillsammans med andra som är vänliga och öppna. Det visar också att du har självförtroendet att säga vad du verkligen tycker -- vilket som du säkert kommer ihåg är en av hörnstenarna i att hävda sig själv. Ge någon en komplimang, redan idag!

\subsubsection{Steg för att ge komplimanger}

\begin{itemize}

\item Tänk ut de exakta orden som du vill använda för att ge komplimangen. Det kommer att göra att du känner dig säkrare, eftersom du slipper söka efter de rätta orden i situationen.

\item Var specifik när du komplimerar den andra personen. ''Det där halsbandet klär dig verkligen'' ger ett större intryck än ''du ser verkligen bra ut idag''. Ju mer specifik du är, desto bättre är det. Det gör att personen känner att du verkligen ser honom eller henne.

\item Mena det du säger, säg det du menar. Människor är ganska bra på att känna skillnaden mellan förställning och ärlighet. 

\item Överdriv inte. Några meningar räcker. (''Du gjorde ett jättebra jobb med \ldots{}'' eller ''Du lyckades verkligen bra med \ldots{}'')
 
\item Le och visa din entusiasm när du ger en komplimang. Det får den andra personen att känna att du verkligen menar det.
 
\item Känn av situationen. Ta hänsyn till platsen ni befinner er på och relationen du har till den andra personen. Att ge en kollega en vänlig kommentar om det nyfärgade håret är helt lämpligt, men att säga samma sak till högsta chefen skulle kunna vara att gå över gränsen.

\item Foga en fråga till din komplimang. Om du vill använda komplimangen som en konversationsöppnare, ställ en fråga till mottagaren av din komplimang: ''Det där halsbandet är verkligen snyggt på dig. Var har du hittat det?''

\end{itemize}

Tänk på ett tillfälle nyligen när du beundrade någon (exempelvis personens kläder, utförda arbete eller goda handlingar) och du \textit{inte} gav henne eller honom en komplimang. Vad skulle du ha velat säga till den personen? Tänk ut vad du skulle kunna ha sagt till honom eller henne, och skriv ned det här:

\xlines{5}

\section{Sammanfattning}

\begin{itemize}

\item Det är en viktig hävda sig själv-färdighet och en viktig social färdighet att kunna ta emot och ge komplimanger.

\item Som med andra färdigheter för konstruktiv självhävdelse kan det finnas hindrande tankar som gör det svårt för dig att ta emot en komplimang på ett vänligt sätt. Du kan utmana och balansera det hindrande tänkandet genom att ge dig tid och struktur för att hitta nya, alternativa tankar.

\item Det är också användbart att kunna visa sin vänlighet mot andra genom att kunna ge komplimanger.

\item Vi behöver alla regelbundet öva på att ta emot och ge komplimanger. Det kan ha en väldigt positiv inverkan på vår egen självkänsla, liksom på självkänslan hos dem vi har omkring oss.

\end{itemize}

\nextup{I nästa kapitel sätter vi samman alla färdigheter vi lärt oss och tittar på hur vi kan börja öva på konstruktiv självhävdelse i vardagen, i våra egna liv.}

\chapter{Knyta ihop säcken}\label{chap10} % korrläst 190420

\section{Sammanfattning av det du lärt dig så här långt}

I den här boken har du fått lära dig vad konstruktiv självhävdelse är för något (kapitel~\ref{chap1}), hur du känner igen passiva, aggressiva och självhävdande beteenden (kapitel~\ref{chap2}), hur du kan balansera tänkande som hindrar konstruktiv självhävdelse (kapitel~\ref{chap3}), en del tekniker för självhävdelse (kapitel~\ref{chap4}) och hur du sänker spänningsnivån i kroppen (kapitel~\ref{chap5}). Du har också tittat på hur du kan hävda dig själv bättre i specifika situationer, till exempel när du vill säga 'nej' (kapitel~\ref{chap6}), när du får kritik eller vill ge kritik (kapitel~\ref{chap7}), när du blir besviken, men vill kunna gå vidare ändå (kapitel~\ref{chap8}) och när du får och vill ge komplimanger (kapitel~\ref{chap9}). 

I det här avslutande kapitlet ska vi sätta samma de här färdigheterna och stegvis visa hur du på egen hand kan förbättra din konstruktiva självhävdelse i ditt vardagsliv.

\section{Steg för att hävda dig själv ännu bättre}

Här är en översikt över stegen mot bättre självhävdande:

\begin{enumerate}

\item Identifiera situationer som du vill jobba med. Tänk på hur du normalt sett hanterar de situationerna. Brukar du vanligtvis använda passiv eller aggressiv kommunikation? Rangordna situationerna, från de lättaste till de svåraste. Se nedan om att upprätta en hierarki.

\item Identifiera hindrande tänkande som är associerat med situationerna du tagit med i listan.

\item Hitta mer konstruktivt tänkande som möjliggör för dig att hävda dig själv i situationen. Använd en tankedagbok om du behöver.

% ordlista?
\item Identifiera hindrande beteende som du brukar använda -- saker som du gör för att undvika eller fly från det jobbiga i situationerna. Kom ihåg att leta efter både verbala och icke-verbala beteenden.

\item Hitta mer konstruktiva beteenden för självhävdelse. Om lämpligt, använd en av teknikerna du lärde dig i kapitel~\ref{chap3}.

\item Öva på vad du ska säga och göra. Det kan ibland hjälpa att skriva ned var du ska säga.

\item Sök upp situationen och handla på det sätt som du har planerat.

\item När du är klar med uppgiften, ge dig själv beröm för det som gick bra och tänk sedan ut vad du behöver förbättra till nästa gång.

\item Fortsätt öva tills du känner dig bekväm med att hävda dig själv konstruktivt i situationen.

\end{enumerate}

\section{Steg 1: upprätta en hierarki}

Börja med att skriva ned tio situationer som du skulle vilja bli bättre på att hävda dig själv i. Det kan vara hemma, på jobbet, med vänner eller ute bland folk.

I kapitel~\ref{skatta} fyllde du i ett formulär med besvärliga situationer (se sidan~\pageref{exercise1}). Det innehåller flera uppslag till situationer som du kan jobba med. Om du inte har fyllt i det formuläret ännu, är det en bra idé att göra det nu. Om du redan har fyllt i det, gå gärna tillbaka och se över det för att påminna dig själv om vad du svarade.

Du kan också ha fått olika idéer om vad du ska jobba med när du fyllde i en tankedagbok i kapitel~\ref{tankedagbok} (se sidan~\pageref{tankedagbok}), eller när du läste kapitlen om särskilda situationer, exempelvis om att säga 'nej', ge eller ta kritik eller hantera besvikelser.

Om du fortfarande kämpar med att identifiera situationer att arbeta med, tänk på följande situationer. Det kan hjälpa dig. Hur brukar du göra när \ldots{}

\begin{itemize}

\item maten du har beställt kommer in kall eller överlagad?

\item någon röker där rökförbud råder?

\item du vill be en vän att betala tillbaka ett lån som han eller hon har tagit av dig?

\item alla lämnar dig ensam med disken?

\item du irriterar dig på en vana hos någon du älskar?

\end{itemize}

Om du märker att du vanligtvis beter dig på ett passivt eller aggressivt sätt i någon av ovanstående situationer, skriv upp den situationen på din lista.

När du har upprättat din lista är det dags att göra en hierarki -- det vill säga att rangordna situationerna från lättast till svårast. Gör en skattning av hur besvärligt du upplever att det är att hävda dig själv varje situation, från 0 till 100. En skattning om ''0'' betyder att situationen inte är svår alls att hävda dig själv i. En skattning om ''100'' betyder att det är den mest ansträngande uppgift du över huvud taget kan tänka dig att utföra. Med dessa referenspunkter kan du tänka ut var situationerna på din lista hamnar på skalan 0--100, från lättast till svårast.

%Här är ett exempel på en hierarki.

\subsection{Exempel på en självhävdelse-hierarki}

{\renewcommand{\arraystretch}{1.8}
\begin{longtable}[]{@{}LDD@{}}
\toprule
\strut\sffamily\bfseries Situation    & \sffamily\bfseries Skattning (0--100)  & \sffamily\bfseries Rang \\
\midrule
\strut\handwriting\Large\raggedright Säga till svärmor att jag inte vill att hon röker inomhus.   &  \handwriting\Large 70  & \handwriting\Large 9 \\
\midrule
\strut\handwriting\Large\raggedright Ringa banken och säga att jag behöver mer tid för att tänka på om jag ska ta lånet eller inte.    &  \handwriting\Large 50  & \handwriting\Large 8 \\
\midrule
\strut\handwriting\Large\raggedright Säga till grannen att deras hund stör min nattsömn.    &  \handwriting\Large 40  & \handwriting\Large 6 \\
\midrule
\strut\handwriting\Large\raggedright Säga till min partner att jag vill ha en kväll i veckan för mig själv.    &  \handwriting\Large 45  & \handwriting\Large 7 \\
\midrule
\strut\handwriting\Large\raggedright Säga förlåt till min kollega för att jag var irritabel i onsdags.    &  \handwriting\Large 30  & \handwriting\Large 3 \\
\midrule
\strut\handwriting\Large\raggedright Säga till barnen att göra sina sysslor hemma.    &  \handwriting\Large 20  & \handwriting\Large 1 \\
\midrule
\strut\handwriting\Large\raggedright Be min vän lämna tillbaka boken jag lånade ut för 3 månader sedan.    &  \handwriting\Large 25  & \handwriting\Large 2 \\
\midrule
\strut\handwriting\Large\raggedright Säga till chefen att jag har för mycket att göra och inte kan vara med i nya projektet.    &  \handwriting\Large 80  & \handwriting\Large 10 \\
\midrule
\strut\handwriting\Large\raggedright Ringa pianostämmaren och säga att pianot fortfarande klingar falskt.    &  \handwriting\Large 40  & \handwriting\Large 5 \\
\midrule
\strut\handwriting\Large\raggedright Säga till pappa hur mycket jag älskar honom.    &  \handwriting\Large 30  & \handwriting\Large 4 \\
\bottomrule
\end{longtable}
}

Börja med den lättaste uppgiften på listan när du är klar med din hierarki. I det här exemplet är den lättaste upptiften att be barnen att sköta sina sysslor i hemmet bättre.

\section{Steg 2 och 3: balansera hindrande tänkande}

Som du såg i kapitel~\ref{chap3} finns det ofta hindrande tankar bakom vår oförmåga att hävda oss själva. Undersök om så är fallet för dig med var och en av uppgifterna på din lista.

För personen i exemplet som hade svårt att säga till barnen, var det här den hindrande tanken: ''Om jag tjatar på barnen att göra sina sysslor kommer de bli irriterade på mig, då kanske de inte tycker om mig längre eller tycker att jag är en dålig mamma.''

Efter att personen använt en tankedagbok hittade hon följande tanke, som stödjer konstruktiv självhävdelse: ''Alla barn blir irriterade på sina mammor ibland, det betyder inte att de inte tycker om sina mammor. Det är viktigt för barn att lära sig hur man gör olika sysslor. För att vara en bra mamma måste jag ibland uppmana barnen att göra saker som de inte gillar. De kanske tackar mig för det senare.''

När personen hade hittat sin nya, balanserande tanke, blev det lättare att hävda sig i situationen, och sedan ta sig an nästa.

\section{Steg 4 och 5: balansera hindrande beteenden}

Personen som skrev hierarkin upptäckte att hon ofta hade skuldkänslor när hon uppmanade barnen att göra sina sysslor, och därför alltid samtidigt bad om ursäkt. Hon brukade dessutom ge dem något de gillar efteråt, som godis eller snabbmat, för att de skulle fortsätta tycka om henne. Hon identifierade detta som hindrande beteenden, som undergrävde hennes önskan att lära barnen att delta i hemarbetet. Hon kom på att det vore ett mer hjälpsamt beteende att helt enkelt bara be barnen att göra sina sysslor, utan att be om ursäkt för det, och att verbalt berömma dem när de var klara.

\section{Steg 6 till 9: repetera och öva}

I det här fallet hade inte personen något behov av att på förhand skriva ned vad hon skulle säga. Hon behövde däremot först rollspela med terapeuten för att öva på att be sina barn att göra sina sysslor, utan att ramla i fällan att samtidigt be om ursäkt. Efter några veckors arbete med att kommunicera rakare -- och utan att köpa barnen något för att kompensera -- började hon känna sig bekvämare.

\section{Nu är det din tur!}

Du har nu läst om vilka stegen är för att öka förekomsten av beteenden som kan beskrivas som konstruktiv självhävdelse. Nu är det din tur att upprätta en egen självhävdelse-hierarki.

\newpage

\subsection{Självhävdelse-hierarki}

{\renewcommand{\arraystretch}{1.8}
\begin{longtable}[]{@{}LDD@{}}
\toprule
\strut\sffamily\bfseries Situation    & \sffamily\bfseries Skattning (0--100)  & \sffamily\bfseries Rang \\
\midrule
\begin{minipage}[c][.07\textheight]{.7\textwidth}\Large\handwriting %1
\end{minipage}  &   &  \\
\midrule
\begin{minipage}[c][.07\textheight]{.7\textwidth}\Large\handwriting %2
\end{minipage}  &   &  \\
\midrule
\begin{minipage}[c][.07\textheight]{.7\textwidth}\Large\handwriting %3
\end{minipage}  &   &  \\
\midrule
\begin{minipage}[c][.07\textheight]{.7\textwidth}\Large\handwriting %4
\end{minipage}  &   &  \\
\midrule
\begin{minipage}[c][.07\textheight]{.7\textwidth}\Large\handwriting %5
\end{minipage}  &   &  \\
\midrule
\begin{minipage}[c][.07\textheight]{.7\textwidth}\Large\handwriting %6
\end{minipage}  &   &  \\
\midrule
\begin{minipage}[c][.07\textheight]{.7\textwidth}\Large\handwriting %7
\end{minipage}  &   &  \\
\midrule
\begin{minipage}[c][.07\textheight]{.7\textwidth}\Large\handwriting %8
\end{minipage}  &   &  \\
\midrule
\begin{minipage}[c][.07\textheight]{.7\textwidth}\Large\handwriting %9
\end{minipage}  &   &  \\
\midrule
\begin{minipage}[c][.07\textheight]{.7\textwidth}\Large\handwriting %10
\end{minipage}  &   &  \\
\bottomrule
\end{longtable}
}

\newpage

\subsection{Arbetsblad för konstruktiv självhävdelse}

Använd det här arbetsbladet med var och en av situationerna i din självhävdelse-hierarki. Det vägleder dig genom alla steg.

\subsubsection{Vilken situation vill jag hävda mig bättre i?}

\xlines{2}

\subsubsection{Vilka hindrande antaganden vidmakthåller problemen?}

\xlines{3}

\subsubsection{Vilka beteenden vore uttryck för konstruktiv självhävdelse?}

\xlines{2}

\subsubsection{Vilka hindrande beteenden använder jag mig av?}

\xlines{2}

\subsubsection{Hur kan jag hävda mig själv mer konstruktivt?}

\xlines{3}

\section{Skatta din självhävdelse igen}

Grattis! Du har nu gått genom alla kapitel i den här arbetsboken. Nu kan du göra om självskattningen från kapitel~\ref{chap1} där du först skattade din förmåga att hävda dig själv i olika situationer och med olika personer. Fyll i formuläret på sidan~\pageref{uppf}, för att se om du har blivit bättre på att hävda dig själv konstruktivt.

Fyll i varje ruta på en skala från 0 till 5. En skattning om ''0''
betyder att du utan problem kan hävda dig själv konstruktivt i den givna situationen.
En skattning om ''5'' betyder att du inte alls kan hävda dig själv konstruktivt i
den situationen.

\section{Sammanfattning}

\begin{itemize}

\item Följ de här stegen för att bli bättre på att hävda dig själv: Skapa först en hierarki med situationer som du vill jobba med, identifiera och balansera sedan hindrande tänkande och beteende i de situationerna. Öva! 

\end{itemize}

\assessment{uppf}

\nextup{När du nu jobbar vidare med konstruktiv självhävdelse i vardagslivet, kom ihåg att övning ger färdighet!}

\backmatter

%%%%%%%%%%%%%%%%%%%%%%%%%%%%%%%%

\nocite{beck1979cognitive}
\nocite{Clark1986}
\nocite{clark1995cognitive}
\nocite{alberti2017your}
\nocite{gambrill1975assertion}
\nocite{ward2018assertiveness}
\nocite{linehan1979assertion}
\nocite{mckay2005self}
\nocite{powell2017mental}
\nocite{smith1975say}
\nocite{wolpe1990practice}

%%%%%%%%%%%%%%%%%%%%%%%%%%%%%%%%


%\cleardoublepage\phantomsection\addcontentsline{toc}{chapter}{Forskningsbakgrund}
%\chapter{Forskningsbakgrund}
\cleardoublepage\phantomsection\chapter{Forskningsbakgrund}

Koncepten och strategierna i den här boken bygger på evidensbaserad psykologpraktik, primärt kognitiv beteendeterapi (KBT). KBT är ett slags psykoterapi som utgår från att hindrande, negativa känslor och beteenden hänger samman med problematisk kognition, det vill säga hindrande tänkande.

%\begin{refsection}

\raggedright
\raggedcolumns

%\bibliographystyle{apacite}
%\bibliography{library}
%\addcontentsline{toc}{chapter}{Forskningsbakgrund}

%\printbibliography[heading=bibintoc,title={Forskningsbakgrund},prenote=wittyquote]

\printbibliography[heading=subbibliography,keyword=arbetsbok,title={Arbetsböcker på svenska}]

\printbibliography[heading=subbibliography,notkeyword=arbetsbok,title={Referenser}]
%\bibliographystyle{apalike}

%\renewcommand\xspecialintrotext{\normalfont\normalsize Litteratur:\par\vspace{.5\baselineskip}}
%\renewcommand\xspecialintrotext{}

%\end{refsection}

\printindex

\end{document}
